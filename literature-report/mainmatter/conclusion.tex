\chapter{Conclusions}
\label{chapter: conclusion}
This literature study sets up an task planning and execution framework without actually implementing it. Many techniques are used in the framework which can best be distinguished in subjects: control methods, system identification and lastly, motion and manipulation planning. These subjects have all been visited, current literature of the research field is categorised, summarised and discussed. Individually, these subjects all have techniques which have been proven to be effective. The proposed framework connects the subjects and claims they can complement each other. \\

After the robot environment has been defined, assumptions are listed an the task was described, the survey starts with answering the first research subquestion.\\

\textit{How are system models learned, and what are the limitations for different system learning techniques.} To fully answer this question, models capable of describing the system models and control methods have been categorised. For \textbf{single-body control, predictive methods and hybrid system models are favoured.} Predictive methods such as \ac{MPC} or \ac{MPPI} can painlessly incorporate constraints, which is essential when controlling nonholonomic robots. Analytical models provide an sensible model at the initial time step, but require too prior knowledge which is unrealistic to assume, additionally analytic models are unable to adapt to system changes. Out of the reviewed system models, data-driven models converge closest to the true dynamics incorporating even small nonlinear effects, but converging takes much \ac{IO} data which takes too much time. Hybrid system models take the best of both worlds, by assuming some structure hybrid systems are able to provide a workable model which converges toward the true dynamics with offline or online adaption. For \textbf{multi-body models predictive methods combined with data-driven approaches are favoured.} Multi-body models are contain primarily nonlinear dynamics which are harder to capture if any assumptions over the true dynamics are taken, this rules out hybrid models and the only viable option are data-driven models. Preferred system identification for single-body control are dominated by \ac{PEM}, with variants specialised for online adaptation or minimal \ac{IO} data required. Multi-body models are best learned with a set of test pushes which are converted to a data-driven model, truth be told, there exist no best system identification methods because it depends on single-/multi-body model, the choice of controller and mainly the situation (environment, robot geometry, object parameters, emphasis on minimal time required or minimal prediction errors). But generally for single-body control use predictive methods with hybrid models, for multi-body control use predictive methods with data-driven models. \\

In \cref{chapter: task_and_motion_planning} the second research subquestion is answered.\\

\textit{Why is task, motion and manipulation planning so much more difficult compared to task and motion planning.} The configuration space becomes piece-wise analytic when adding manipulation planning to motion planning. Such a piecewise-analytic configuration space operates in different modes of dynamics. For example driving or pushing, the dynamical model for both verbs is different. This causes discontinuities in the configuration space at the boundaries where two different modes of dynamics meet. Planning in such a piece-wise analytic configuration space causes an combinational explosion of possibilities, because pushing an object now influences how the configuration space looks like in the future for the mode of driving. The configuration spaces dimensionality grows linearly with the number of objects in it, which even for sampling based planners is not computationally feasible. As addition to motion planning, manipulation planning finds extra constraints to satisfy due to a generally more complex dynamical model. Most changes between motion and manipulation planning reside in the local planner, which validates if 2 sample configurations are connectable with a system model. Task planning has been shown to be NP-hard by comparing it to the standards piano's mover problem, which is known to be NP-hard. Luckily the goal is not to solve a NP-hard problem, but to learn dynamical models during execution of actions. \\

\Cref{chapter: proposed_method} answers the last subquestion partly.\\

\textit{Can planning in configuration space be extended to a piece-wise analytic configuration space with learned dynamics.} Next to recent literature answering the subquestion with a yes \cite{sabbagh_novin_model_2021}, the answer is in the form of a proposed method. Anwering the research subquestion mediocrily. There are no proofs given, tests performed or even an implementation of the proposed method. However the components used by the proposed method individually have established guarantees. The proposed method contains the hypothesis graph, which appears to be an robust algorithm after visual inspection, as does the knowledge graph. The proposed methods claims to solve \ac{NAMO} problems, with the ability to push movable obstacles to target positions while learning object dynamics. To reinforce the claims that the proposed methods makes, benchmark tests have been listed. Expected outcomes to the benchmark test have been hardcoded, which is for the final thesis to actually implement and compare against expected outcomes. The subquestion can be answered with yes, it is possible to extend a configuration space to a piece-wise analytic configuration space with learned dynamics. The answer can be given only because of the citation given \cite{sabbagh_novin_model_2021}, the proposed method on its own would not be sufficient. \\

Lastly, the main research question. \\

\textit{Can objects’ system models be learned by a robot during task execution, and can these newly learned models improve task, motion and manipulation planning?} Maybe, the proposed method gives a promising algorithm, and gives promises with benchmark test what is might accomplish. Every benchmark test clearly indicates how the proposed methods accomplishes results which existing literature would not be able to accomplish. 


