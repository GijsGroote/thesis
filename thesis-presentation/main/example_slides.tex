

\section{Examples}
\begin{frame}{Example frame 2}
  \begin{block}{Block}
    \begin{itemize}
      \item item 1
      \item item 2
    \end{itemize}
  \end{block}

  \begin{exampleblock}{Example}
    \begin{enumerate}
      \item Sugar in a stirred cup of tea gathers in the middle.
      \item Rivers often take a detour through flat terrain.
    \end{enumerate}
  \end{exampleblock}

  \begin{alertblock}{Alert}
     Rivers and sweet tea do unexpected things.\cite{_unimate_2022} 
   \end{alertblock}
\end{frame}

\begin{frame}{Mass--energy equivalence}
	They say every formula you add to a presentation, will reduce your audience by \SI{50}{\percent}. A simple yet effective way to mitigate this effect, is adding a compact nomenclature to the slides containing formulae.
	
	\[E=mc^2\]
	
	If you find this is taking up too much of your precious space, than you are doing something wrong, and it is not adding this little nomenclature.
	
	The optional argument specifies the number of column pairs.
	
\mininomen[2]{% number of columns
  $E$ & Energy (\unit{J})                     & $m$ & Mass (\unit{kg}) \\
  $c$ & Speed of light in vacuum (\unit{m/s}) \\[2ex] % may need some tweaking
  }
\end{frame}

\begin{frame}{columns}
  \begin{columns}[onlytextwidth]
    \begin{column}{.5\textwidth}
      first column
    \end{column}
    \begin{column}{.5\textwidth}
      % square filling the column
      \textcolor{tudCyan}{\rule{1\columnwidth}{1\columnwidth}}
      % place an image
      % horizontal position = 73%
      % vertical position = 45%
      % width = 40% of page
      \absimage{.73, .45}{.40}{figures/tud-logo.pdf}
    \end{column}
  \end{columns}
\end{frame}

\section{Conclusion}
\begin{frame}[fragile]{animation}
  \vfill
  Some commands take optional arguments in the form of \verb|<x-y>|,
  where \verb|x| is the first `sub-frame' on which the context is shown,
  and \verb|y| is the last. \verb|x| or \verb|y| can be replaced by \verb|+|,
  referring to `the next sub-frame'. 
  \vfill
  \begin{columns}[onlytextwidth]
  \begin{column}{.5\textwidth}
    \begin{enumerate}
      \item<+-> uncovered\ldots
      \item<+-> one\ldots
      \item<+-> by\ldots
      \item<+-> one.
    \end{enumerate}
    \end{column}
  \begin{column}{.5\textwidth}
      Using only:\only<1>{1}\only<2>{2}\only<3>{3}

      Using onslide:\onslide<1>{1}\onslide<2>{2}\onslide<3>{3}

      Using pause:\pause1\pause2\pause3
  \end{column}
  \end{columns}
  \vfill
  For more advanced animations, see \S 14 of the manual:\\
  \url{https://www.ctan.org/pkg/beamer}
  % \url{https://www.ctan.org/pkg/animate}\\
  % \url{https://www.ctan.org/pkg/media9}
  \vfill
  % \transduration{2} automatic progression of slides
  \transpush<1>
\end{frame}

\begin{frame}
  Thanks for your attention.

  A digital version of this presentation can be found here:
  \vfill
  \url{https://gitlab.com/novanext/tudelft-beamer} 
  \vfill  
  \centering
  \qrcode{https://gitlab.com/novanext/tudelft-beamer}
  \vfill
\end{frame}


\begin{frame}[allowframebreaks,t]{\bibname}
	% the 'I' is caused by 'allowframebreaks'
	\AtNextBibliography{\footnotesize}% or in the preamble \AtBeginBibliography{\small}
	\printbibliography
\end{frame}


