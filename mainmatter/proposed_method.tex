\chapter{The Hypothesis and Knowledge graph}
\textit{
As defined in the introduction, a task consists of obstacles with corresponding target positions, an obstacle with an accompanying target position is called a subtask. The \textbf{hypothesis graph} contains nodes corresponding to obstacles at a certain position and edges corresponding to actions which relocate obstacles, and connect nodes in the hypothesis graph. A formal description of the hypothesis graph with its nodes and edges is given in \cref{subsec: hgraph_definition} and \cref{subsec: sys_iden_and_control_methods} elaborates the meaning of actions taken when traversing over one of the hypothesis graphs edges.\\
\newline
For every subtask in a task, a start and a target node is created, the hypothesis graph tries to connect the starting node to the corresponding target node by adding nodes and edges. The choice and content of these nodes and edges is based on local planning and randomisation, elaborated in \cref{subsec: estimating_path_existence,subsec: motion_planning,subsec: manipulation_planning}. A successive path from a starting node to the corresponding target node is called a hypothesis. During the search for a hypothesis, the hypothesis graph resides in the \textbf{search loop}, when traversing over edges toward the target node the hypothesis graph resides in the \textbf{execution loop}, both loops are elaborated in \cref{subsec: 2_loops}. When the hypothesis graph traverses over an edge, fault detectors monitors the progress, elaborated in \cref{subsec: fault_detection}. Finally an example of an hypothesis graph is given in \cref{subsec: hgraph_example}.\\
\newline
After execution, the traversed edge enters a review period, during which performance is checked in various metrics. The review is stored in a database, called the \textbf{knowledge graph} which serves to collect edges and rank them based on performance. The knowledge graph is defined in \cref{subsec: kgraph_definition}, metrics to rank edges can be found in \cref{subsec: edge_metrics} and an example is displayed in \cref{subsec: kgraph_example}.
}

\newpage

\section{Hypothesis Graph}
\subsection{Definition}
\label{subsec: hgraph_definition}

\todo[inline]{definition of state, node, }



\textbf{Lifetime and types of edges}
\newline
An edge denotes that input commands are sent to the robot and as a result the robot can drive around changing it's state. Obstacles in a certain state represented by a node in the hgraph can be connected to other nodes via edges.
The edge contains all necessary components to sent input to the robot resulting in an obstacle reaching the target state residing in the node where the edge points toward. \\

There exist two types of edges, identification edges and action edges. An identification edge is responsible for sending an input sequence to the system and recording the system output which the simulation returns. An input/output sequence and assumptions on the system are the basis for system identification techniques discussed in \cref{subsec: sys_iden_and_control_methods}. The goal is to create a dynamical model which if augmented with an corresponding controller is closed-loop stable. \\

An action edge is responsible for putting an obstacle at a target state, the goal is to reach the target state, without triggering the fault detector and scoring well at internal metrics elaborated in \cref{subsec: edge_metrics}.

\todo[inline]{the lifetime of an identification edge}

The different status an edge can have is listed below. An edge status can only change according to the directed arrows in \cref{tikz: lifetime_action_edge}

\definecolor{lavenderIndigo}{RGB}{160, 110, 224}
\definecolor{batteryChargedBlue}{RGB}{22, 164, 216}
\definecolor{skyBlue}{RGB}{96, 219, 232}
\definecolor{kiwi}{RGB}{139, 211, 70}
\definecolor{minionYellow}{RGB}{239, 223, 72}
\definecolor{deepSaffron}{RGB}{249, 165, 44}
\definecolor{sinopia}{RGB}{214, 78, 18}

\begin{figure}[H]
\centering
\begin{tikzpicture}[node distance = 2cm, auto]
    \node [block, fill=lavenderIndigo] (init) {Initialised};
    \node [block, fill=batteryChargedBlue, below of=init] (path_exist) {Path Exists};
    \node [block, fill=skyBlue, below of=path_exist] (system_model) {System Model};
    \node [block, fill=kiwi, below of=system_model] (path_planned) {Path Planned};
    \node [block, fill=minionYellow, below of=path_planned] (executing) {Executing};
    \node [block, fill=deepSaffron, below of=executing] (completed) {Completed};
    \node [block, fill=sinopia] (failed) at ([xshift=4cm]$(system_model)!0.5!(path_planned)$) {Failed};
    
    % arrows
    \draw [-stealth] ([xshift=-2cm]init.west) to node[near start,above]{select controller} (init.west);
    \draw [-stealth] (init.south) to node[left]{graph-based path estimation} (path_exist.north);
    \draw [-stealth] (path_exist.west) [out=215,in=145] to node[left]{system identification} ([yshift=0.3cm] system_model.west);
    \draw [-stealth] ([yshift=-0.3cm] system_model.west) [out=215,in=145] to node[left]{motion planning} ([yshift=0.3cm] path_planned.west);
    \draw [-stealth] ([yshift=-0.3cm] path_planned.west) [out=215,in=145] to  node[left]{go to execution loop} ([yshift=0.3cm] executing.west);
    \draw [-stealth] (executing.south) to node[left]{evaluate with metrics} (completed.north);
    % edges to failed
    \draw [-stealth] (init.east) [out=0, in=90] to node[right]{path non-existence proven}  ([xshift=0.3cm]failed.north);
    \draw [-stealth] (path_exist.east) [out=0, in=90] to node[xshift=-0.4cm,yshift=0.55cm, above]{\shortstack[l]{system\\identification\\error}}  ([xshift=-0.3cm]failed.north);
    \draw [-stealth] (system_model.east) [out=0, in=180] to node[xshift=0.1cm, yshift=0.3cm, above]{\shortstack[l]{motion\\planning\\error}} ([yshift=0.3cm]failed.west);
    node[right]{motion planning error}  
    ([yshift=-0.3cm]failed.west);
    \draw [-stealth] (executing.east) [out=0, in=-90] to node[right]{fault detected}(failed.south);
\end{tikzpicture}
\caption{The status of an action edge and the most important indicator to change status}
\label{tikz: lifetime_action_edge}
\end{figure}

\begin{enumerate}
    \item[INITIALISED] The edge is created with a source and target node which are present in the hypothesis graph. A choice of controller is made.
    \item[PATH EXISTS] A graph-based search is performed to validate if the target state is reachable assuming that the system is holonomic.
    \item[SYSTEM MODEL] A dynamics system model is provided to the controller residing in the edge.
    \item[PATH PLANNED] Resulting from a sampble-based planner, a path from start to target state is provided. 
    \item[EXECUTING] The edge is currently sending input toward the robot. 
    \item[COMPLETED] The edge has driven the system toward it's target state and it's performance has been calculated.
    \item[FAILED] An error occurred, yielding the edge unusable. 
\end{enumerate}

\subsection{System Identification and Control Methods}
\label{subsec: sys_iden_and_control_methods}

\subsection{Estimating Path Existence}
\label{subsec: estimating_path_existence}
\subsection{Motion Planning}
\label{subsec: motion_planning}
\subsection{Manipulation Planning}
\label{subsec: manipulation_planning}
\subsection{The Search and the Execution loop}
\label{subsec: 2_loops}
\subsection{Fault Detection}
\label{subsec: fault_detection}
\subsection{Example}
\label{subsec: hgraph_example}

\section{Knowledge Graph}
\subsection{Definition}
\label{subsec: kgraph_definition}
\subsection{Edge Metrics}
\label{subsec: edge_metrics}
\subsection{Example}
\label{subsec: kgraph_example}
