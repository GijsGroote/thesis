\chapter{The Hypothesis and Knowledge graph}
\textit{
  As defined in the introduction, a task consists of obstacles with corresponding target positions, an obstacle with an accompanying target position is called a subtask, termonology used is presented in \cref{table:proposed_method_termonology}. The \textbf{hypothesis graph} contains nodes corresponding to obstacles at a certain state, edges correspond to actions which changing the obstacle's state, and connect nodes in the hypothesis graph. A formal description of the hypothesis graph with its nodes and edges is given in \cref{subsec:hgraph_definition}, \cref{subsec:sys_iden_and_control_methods} gives more details on the meaning of traversing over edges in the hypothesis graphs. \Cref{tikz:flowchart_proposed_method} presents a schematic overview of the interconnection of the knowledge-, hypothesis graph and the robot environment.\bs
}

\begin{figure}[H] \centering
\begin{tikzpicture}[node distance = 2cm, auto]
    % Place nodes
    \node[draw=gray, rounded corners, inner sep=3ex, line width=7pt, fill=gray, fill opacity=0.4, minimum height=12.1cm, minimum width=6.8cm, yshift=3.7cm] (focusbox) {}; 
    \node[yshift=5.5cm, xshift=-0.6cm, align=left] at (focusbox) {\textbf{Thesis focus}};
   
   \node [outer sep=0cm] (environment) at (0,0)  {\includegraphics[width=5.0cm]{figures/example_environment_cropped.png}}; 

   \node [below, xshift=0.2cm, yshift=-.1cm, text width=5cm, align=left, outer sep=0cm] at (environment.north) {\textbf{Robot Environement}};
   
    \draw [myEvenLighterColor,
    rounded corners=0.3cm, 
    line width=0.3cm]  
    (environment.north west) -- 
    (environment.north east) --
    (environment.south east) --
    (environment.south west) -- cycle  ;
    
    \node [block,
    above of=environment,
    minimum height=2cm,
    minimum width=5cm,
    node distance=4.3cm,
    outer sep=0cm] (hgraph) {Hypothesis Algorithm};
   
    \node [block, 
    above of=hgraph, 
    node distance=3.5cm, 
    minimum width=5cm,
    minimum height=2.0cm] (kgraph) {Knowledge Graph};
      
    \node [rectangle, draw, 
    fill=myEvenLighterColor, 
    text width=5em, text centered, rounded corners, 
    right of=kgraph, 
    minimum width=4cm,
    minimum height=2cm,
    node distance=8cm] (ontology) {Ontology};
     
    \node [rectangle, draw, 
    fill=myEvenLighterColor, 
    text width=5em, text centered, rounded corners, 
    right of=hgraph, 
    minimum width=4cm,
    minimum height=2cm,
    node distance=8cm] (planner) {High-level planner};
    
    % Draw edges
    \draw[-stealth] ([yshift=0.155cm, xshift=0.4 cm]environment.north) -- node [right] {\shortstack[]{sensor\\measurements}}([xshift=0.4 cm]hgraph.south) ;
    \draw[-stealth] ([xshift=-0.4 cm]hgraph.south) -- node [left] {robot input}([yshift=0.155cm, xshift=-0.4 cm]environment.north) ;
    \draw[-stealth] (planner.west) -- node [pos=0.37, above] {task}(hgraph.east);
    \draw[-stealth] ([xshift=-0.4cm]kgraph.south) -- node [left] {\shortstack[]{action\\suggestions}}([xshift= -0.4cm]hgraph.north) ;
    \draw[stealth-] ([xshift=0.4cm]kgraph.south) -- node [right] {action feedback}([xshift= 0.4cm]hgraph.north) ;
    \draw[-stealth] (kgraph.east) -- node [above, pos=0.63] {\shortstack[]{environment\\knowledge}}(ontology.west);
    \draw[stealth-] ([xshift=0.4cm]ontology.south) -- node [right] {\shortstack[]{query}}([xshift=0.4cm]planner.north);
    \draw[-stealth] ([xshift=-0.4cm]ontology.south) -- node [left] {\shortstack[]{output}}([xshift=-0.4cm]planner.north);
    \draw[stealth-] (planner.south) |- ++ (2,-1) node[near end, above] {\shortstack[]{High-level\\task}};
    \end{tikzpicture}
\caption{Simplified flowchart representation of the proposed method. This thesis work could be augmented with an ontology and high-level planner as displayed. Such an augmentation would create a framework capable of completing high-level tasks.}
\label{tikz:flowchart_proposed_method}
\end{figure}

\todo[inline]{Review/restyle/update the flowchart here above}

\begin{table}[ht]
\centering
\begin{tabular}[t]{l p{10cm}}
  Task:& Collection of obstacles and target states, defined in \cref{subsec:task}\\
Subtask:& a single obstacle, and a single target state \\
Edge:& transition in the todod this here \\
Hypothesis:& idea, sequence of edges to complete a subtask\\
\end{tabular}
\caption{Terminology of terms used}
\label{table:proposed_method_termonology}
\end{table}

\section{Hypothesis Graph}
\label{sec:hgraph}
\todo[inline]{this should be rewritten as hgraph intro}
For every subtask in a task, a start and a target node is created, the hypothesis graph tries to connect the starting node to the corresponding target node by adding node and edges. The choice and content of these nodes and edges is based on local planning and randomisation, elaborated in~\cref{subsec:estimating_path_existence,subsec:motion_planning,subsec:manipulation_planning}. A successive path from a starting node to the corresponding target node is called a hypothesis. During the search for a hypothesis, the hypothesis graph resides in the \textbf{search loop}, when traversing over edges toward the target node the hypothesis graph resides in the \textbf{execution loop}, both loops are elaborated in \cref{subsec:2_loops}. When the hypothesis graph traverses over an edge, fault detectors monitors the progress, elaborated in \cref{subsec:fault_detection}. Finally an example of an hypothesis graph is given in \cref{subsec:hgraph_example}.\bs

After execution, the traversed edge enters a review period, during which performance is checked in various metrics. The review is stored in a database, called the \textbf{knowledge graph} which serves to collect edges and rank them based on performance. The knowledge graph is defined in \cref{subsec:kgraph_definition}, metrics to rank edges can be found in \cref{subsec:edge_metrics} and an example is displayed in \cref{subsec:kgraph_example}.
\todo[inline]{come back to joint configuration space, and tell the hgraph cuts the different modes of dynamics with an edge for every mode of dynamics}


\subsection{Definition}
\label{subsec: hgraph_definition}

\todo[inline]{definition of state, node, }



\textbf{Lifetime and types of edges}
\newline
An edge denotes that input commands are sent to the robot and as a result the robot can drive around changing it's state. Obstacles in a certain state represented by a node in the hgraph can be connected to other nodes via edges.
The edge contains all necessary components to sent input to the robot resulting in an obstacle reaching the target state residing in the node where the edge points toward. \\

There exist two types of edges, identification edges and action edges. An identification edge is responsible for sending an input sequence to the system and recording the system output which the simulation returns. An input/output sequence and assumptions on the system are the basis for system identification techniques discussed in \cref{subsec: sys_iden_and_control_methods}. The goal is to create a dynamical model which if augmented with an corresponding controller is closed-loop stable. \\

An action edge is responsible for putting an obstacle at a target state, the goal is to reach the target state, without triggering the fault detector and scoring well at internal metrics elaborated in \cref{subsec: edge_metrics}.

\todo[inline]{the lifetime of an identification edge}

The different status an edge can have is listed below. An edge status can only change according to the directed arrows in \cref{tikz: lifetime_action_edge}

\definecolor{lavenderIndigo}{RGB}{160, 110, 224}
\definecolor{batteryChargedBlue}{RGB}{22, 164, 216}
\definecolor{skyBlue}{RGB}{96, 219, 232}
\definecolor{kiwi}{RGB}{139, 211, 70}
\definecolor{minionYellow}{RGB}{239, 223, 72}
\definecolor{deepSaffron}{RGB}{249, 165, 44}
\definecolor{sinopia}{RGB}{214, 78, 18}

\begin{figure}[H]
\centering
\begin{tikzpicture}[node distance = 2cm, auto]
    \node [block, fill=lavenderIndigo] (init) {Initialised};
    \node [block, fill=batteryChargedBlue, below of=init] (path_exist) {Path Exists};
    \node [block, fill=skyBlue, below of=path_exist] (system_model) {System Model};
    \node [block, fill=kiwi, below of=system_model] (path_planned) {Path Planned};
    \node [block, fill=minionYellow, below of=path_planned] (executing) {Executing};
    \node [block, fill=deepSaffron, below of=executing] (completed) {Completed};
    \node [block, fill=sinopia] (failed) at ([xshift=4cm]$(system_model)!0.5!(path_planned)$) {Failed};
    
    % arrows
    \draw [-stealth] ([xshift=-2cm]init.west) to node[near start,above]{select controller} (init.west);
    \draw [-stealth] (init.south) to node[left]{graph-based path estimation} (path_exist.north);
    \draw [-stealth] (path_exist.west) [out=215,in=145] to node[left]{system identification} ([yshift=0.3cm] system_model.west);
    \draw [-stealth] ([yshift=-0.3cm] system_model.west) [out=215,in=145] to node[left]{motion planning} ([yshift=0.3cm] path_planned.west);
    \draw [-stealth] ([yshift=-0.3cm] path_planned.west) [out=215,in=145] to  node[left]{go to execution loop} ([yshift=0.3cm] executing.west);
    \draw [-stealth] (executing.south) to node[left]{evaluate with metrics} (completed.north);
    % edges to failed
    \draw [-stealth] (init.east) [out=0, in=90] to node[right]{path non-existence proven}  ([xshift=0.3cm]failed.north);
    \draw [-stealth] (path_exist.east) [out=0, in=90] to node[xshift=-0.4cm,yshift=0.55cm, above]{\shortstack[l]{system\\identification\\error}}  ([xshift=-0.3cm]failed.north);
    \draw [-stealth] (system_model.east) [out=0, in=180] to node[xshift=0.1cm, yshift=0.3cm, above]{\shortstack[l]{motion\\planning\\error}} ([yshift=0.3cm]failed.west);
    node[right]{motion planning error}  
    ([yshift=-0.3cm]failed.west);
    \draw [-stealth] (executing.east) [out=0, in=-90] to node[right]{fault detected}(failed.south);
\end{tikzpicture}
\caption{The status of an action edge and the most important indicator to change status}
\label{tikz: lifetime_action_edge}
\end{figure}

\begin{enumerate}
    \item[INITIALISED] The edge is created with a source and target node which are present in the hypothesis graph. A choice of controller is made.
    \item[PATH EXISTS] A graph-based search is performed to validate if the target state is reachable assuming that the system is holonomic.
    \item[SYSTEM MODEL] A dynamics system model is provided to the controller residing in the edge.
    \item[PATH PLANNED] Resulting from a sampble-based planner, a path from start to target state is provided. 
    \item[EXECUTING] The edge is currently sending input toward the robot. 
    \item[COMPLETED] The edge has driven the system toward it's target state and it's performance has been calculated.
    \item[FAILED] An error occurred, yielding the edge unusable. 
\end{enumerate}

\subsection{System Identification and Control Methods}
\label{subsec:sys_iden_and_control_methods}

\subsection{Estimating Path Existence}
\label{subsec:estimating_path_existence}

Graph-based planning

\quotes{The main idea is to discretise the configuration space with a finite discretisation. The emerged cells act as nodes in the graph, cells are connected through edges to nearby cells. Graph-based planners start from the cell containing the starting pose and search for the cell containing the target pose whills avoiding cells which lie in obstacle space.}
\todo[inline]{check de quote hierboven}



\subsection{Motion Planning}
\label{subsec:motion_planning}

In order to place an obstacle at a new location, an path connecting the start to target pose is required, a practical example of such an path is a list of successive robot poses, from starting pose in small steps (reachable for the robot in $\sim10$ time samples) toward the target pose. The problem to find a path the robot can track to drive from a start- to target pose whils avoiding obstacles is referred to as \textit{motion planning}. Finding a path between start and target pose for pushing application avoiding collision is referred to as \textit{manipulation planning}. First this subsection presents motion planning, next section dedicates itself to manipulation planning. To obtain a discrete path sampling-based methods are used, described as.\bs

\textit{\quotes{The main idea is to avoid the explicit construction of the obstacle spac, and instead conduct a search that probes the configuration space with a sampling scheme. This probing is enabled by a collision detection module, which the motion planning algorithm considers as a “black box.”~\cite{lavalle_planning_2006}}}\bs

For this purpose a dedicated motion planning algorithm has been developed. The developed motion planning algorithm extends the existing double tree \ac{RRT*} algorithm~\cite{chen_fast_2018}. The modification allows the motion planning algorithm to detect known or unknown obstacles. \ac{RRT*} keeps track of a cost to source node in order to find the shortest path, a fixed panalty is given to paths crossing through known or unknown obstacles as displayed in~\cref{fig:double_rrt_alg}. What can be seen in~\cref{fig:double_rrt_alg} is that the motion planner cannot find a path around the movable obstacle and is forced to add the cost to move the obstacle. If a path around movable or unknown obstacles exist the motion planner finds it with infinite sampling. The added fixed cost for a path crossing through a movable or unknown obstacles motivates the motion planner to find the shortest path around obstacles but prefers moving an obstacle over making a large detour.

\begin{figure}[H]
    \centering
    \includegraphics[width=0.9\textwidth]{figures/rrt_with_costs.png}
    \caption{Schematic view of the proposed double $\text{RRT}^*$ tree taking movable and unknown obstacles into account with cost to reach a sampled configuration displayed.}
    \label{fig:double_rrt_alg}
\end{figure}
\todo[inline]{Update the figure above}


\cref{pseudocode:proposed_rrt_star} displays the pseudocode for the proposed modified \ac{RRT*} algorithm which takes unknown and movable obstacles into account. The following definitions are used by the proposed algorithm.\bs


\begin{center}
\begin{tabular}[t]{l p{10cm}}
$x$:& A node containing a point in configuration space, a flag indicating if the node in in the start tree or target tree and the cost toward initial node\\
$V$:& A set of nodes\\
$E$:& A set of edges\\
$P$:& A set of paths\\
\end{tabular}
\end{center}
\todo[inline]{Is this too vague? the paths could be defined more specifically}


The following functions are called by the \cref{pseudocode:proposed_rrt_star}.\\ 

\begin{center}
\begin{tabular}[t]{l p{10cm}}
$x_{init}$:& Creates a start and target node\\ 
$NotReachStop$:& True if the stopping criteria is not reached\\ 
$Sample_{random}$:& Creates a random sample in free-, movable- or unknown space\\
$Nearest(x, V)$:& Returns the nearest nodes from $x$ in $V$\\
$NearestSet(x, V)$:& Returns set of nearest nodes from $x$ in $V$\\
$Project(x, x')$:& Project $x$ toward $x'$\\
$CollisionCheck(x)$:& Returns true if $x$ is in free-, movable- or unknown space\\
$ObstacleCost(x', x)$:& Returns a fixed additional cost if $x$ enters movable- or unknown space from $x'$, otherwis returns 0\\
$Distance(x, x')$:& Returns the distance between sample $x$ and $x'$\\
$CostToInit(x)$:& Find the total cost from $x$ to the initial node\\
$LocalPlannerCheck(x, x')$:& Return true if a local planner is able to connect $x$ and $x'$, otherwise return false \\
$InSameTree(x, x')$:& Returns true if both $x$ and $x'$ are in the same tree, otherwise return false\\
\end{tabular}
\end{center}

\todo[inline]{make 3 images of motion planning}
\todo[inline]{create 3 sections, connect those with a lil explainer and the 3 imagase}

\begin{algorithm}[H]
\caption{Pseudocode for modified $\text{RRT}^*$ algorithm taking movable obstacles and constraints into account}
\label{pseudocode:proposed_rrt_star}
\begin{algorithmic}[1]
\State $V \leftarrow x_{init}$
\While{$NotReachStop$} 
    \State $Cost_{min} \leftarrow +\infty$ \algorithmiccomment{Create and check a new random sample}
    \State $x_{rand} \leftarrow Sample_{random}$
    \State $x_{nearest} \leftarrow Nearest(x_{rand}, V)$
    \State $x_{temp} \leftarrow Project(x_{rand}, x_{nearest})$
    \If{$CollisionCheck(x_{temp})$}
        \State $x_{new} = x_{temp}$
        \Else
        \State $Continue$
    \EndIf
    \bs 
    \State $X_{near} \leftarrow NearestSet(x_{new}, V)$ \algorithmiccomment{Add new node to the parent node\newline yielding lowest total cost }
    \For{$x_{near} \in X_{near}$}
    \State $Cost_{temp} \leftarrow CostFromInit(x_{near}) + Distance(x_{near}, x_{new}) + ObstacleCost(x_{near}, x_{new})$
    \If{$Cost_{temp}  < Cost_{min}$}
            \If{$LocalPlannerCheck(x_{new}, x_{near})$}
            \State $Cost_{min} \leftarrow x_{temp}$
            \State $x_{minCost} \leftarrow x_{near}$
            \EndIf
        \EndIf
    \EndFor
    \If{$Cost_{min} == \infty$}
        \State $Continue$
    \Else
        \State $V.add(x_{new})$
        \State $E.add(x_{minCost}, x_{new})$
    \EndIf

    \bs
    \State $Cost_{path} \leftarrow +\infty$ \algorithmiccomment{Check if newly added node can lower cost for nearby nodes}
    \For{$x_{near} \in X_{near}$} 
      \If{$InSameTree(x_{near}, x_{new})$} 
        \State $Cost_{temp} \leftarrow CostFromInit(x_{new}) + distance(x_{new}, x_{near}) + ObstacleCost(x_{new}, x_{near})$
        \If{$Cost_{temp} < CostFromInit(x_{near})$}
           \If{$LocalPlannerCheck(x_{new}, x_{near})$}
              \State $E.rewire(x_{near}, x_{new})$
           \EndIf
        \EndIf
      \Else \algorithmiccomment{Add lowest cost path to list of paths}
          \State $Cost_{temp} \leftarrow CostFromInit(x_{new}) + distance(x_{new}, x_{near}) $ \newline\hspace*{10em} $+ CostFromInit(x_{near}) + ObstacleCost(x_{new}, x_{near})$
          \If{$Cost_{temp}  < Cost_{path}$}
              \If{$LocalPlannerCheck(x_{new}, x_{near})$}
                  \State $Cost_{pathMin} \leftarrow x_{temp}$
                  \State $x_{pathMin} \leftarrow x_{near}$
              \EndIf
          \EndIf
      \EndIf
      \If{$Cost_{pathMin} == \infty$}
          \State $Continue$
      \Else
          \State $P.addPath(x_{new}, x_{pathMin}, Cost_{pathMin})$
      \EndIf
    \EndFor
\EndWhile
\end{algorithmic}
\end{algorithm}


\todo[inline]{connect localplannercheck to dynamical models}
\todo[inline]{do you want examples? graph with big-o times? practical stuff sucha as if it takes more than 0.5 sec -> halt without a path.}






% should these subsections not belong in some subdirectory gijs?
\subsection{Manipulation Planning}
\label{subsec:manipulation_planning}
\subsection{The Search and the Execution loop}
\label{subsec:2_loops}
\subsection{Fault Detection}
\label{subsec:fault_detection}
\subsection{Example}
\label{subsec:hgraph_example}

\section{Knowledge Graph}
\label{sec:kgraph}
\todo[inline]{this should get a intro, what does that kgraph dooo}
\subsection{Definition}
\label{subsec:kgraph_definition}


\todo[inline]{this section}

\subsection{Edge Metrics}
\label{subsec:edge_metrics}
\subsection{Example}
\label{subsec:kgraph_example}



