\chapter{Introduction}
\label{chap: introduction}

\todo[inline]{citation for claim "3 main topics"}

Three main topics in robotics are motion planning, placing objects at target locations, and learning obstacle dynamics. Whilst individually a lot of research is done in these topics, combining all three is sparsely researched. This thesis reports proposes an robotic framework which can learn obstacle dynamics, perform motion planning and place obstacles at target positions. 


\todo[inline]{3 citations, for motion planning, placing target positions, and one for learning obstacle dynamics}

\todo[inline]{citation for combi learing and placing at target position, this one \cite{sabbagh_novin_model_2021}}

\todo[inline]{citation for combi of learnign and  motion planning, this one \cite{scholz_navigation_2016}}

\todo[inline]{citation for combi of motion planning and placing at target location, this one \cite{goldberg_asymptotically_2020}}

\todo[inline]{citation for combining all 3, + tell that this will be something which we will compare against, this one \cite{sabbagh_novin_model_2021}}



For the starting bit in the intro:\\
- big problem: field domain, what field knowns\\
- narrow problem within\\
- my approach and the gap it fills\\


With references, explain the current state of knowledge, identify the gap in knowledge to fill 

\section{Research Question}

\textbf{Main research question:}
\begin{center}
\label{researchquestion: main}
\large
How do objects' system models learned by a nonprehensile manipulation robot during\\  task execution improve global task planning?
\end{center} 

\textbf{Research subquestion:}
\begin{enumerate}
    \item \label{researchsubquestion: does_it_work} How does backtracing \cite{krontiris_dealing_2015} while remembering interactions with unknown obstacles compare to not remembering learning interactions over time?
    \item\label{researchsubquestion: does_it_compare} Can the proposed method combine learning and planning for push en drive applications? Can the proposed method complete tasks, and how does it compare against the state of the art? 
\end{enumerate}

See research question \ref{researchquestion: main}
\todo[inline]{make research question referable}
\cref{researchsubquestion: does_it_compare}

\section{Problem Descriptions}
\todo[inline]{your assumptions were lacking during the midterm meeting, clearify your assumptions, especially quasi static and the shapes of obstacles}
\section{Report Structure}