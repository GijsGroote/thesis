\chapter{Conclusions}%
\label{chap:conclusion}
\textit{This chapter todo A.\bs}

Recall the \hyperref[researchquestion:main]{main research question}. \vspace{0.5\baselineskip}\\
\textit{\indent\quotes{How do objects' system models learn by a nonprehensile manipulation robot during task \\ \indent execution improve global task planning?}}\vspace{\baselineskip}

\noindent Before answering the main research question, the two research subquestions are answered.\vspace{\baselineskip}

\noindent Recall \hyperref[researchsubquestion:does_it_work]{first research subquestion}:\vspace{0.5\baselineskip}\\
\textit{\indent\quotes{How does a backward search~\cite{krontiris_dealing_2015} while remembering interactions with unknown objects compare to\\\indent not remembering learning interactions over time?}}\vspace{0.5\baselineskip}

\noindent The backwards search technique used by the hypothesis graph has been shown to successfully handle uncertain action sequences toward task completion. Replanning occurs when edges fail to complete. To obtain a clear overview of the effect of the learned system model, the effectiveness of the hgraph has been monitored over time in various metrics. Two cases have been researched, task execution whilst remembering interaction in the knowledge graph versus task execution without a knowledge graph. The main difference was the time difference because the system identification procedure was executed for every edge in the no KGraph case. Another difference was spotted, over time the KGraph case generates hypothesises which completed more often compared to the no KGraph case. \vspace{\baselineskip}


\noindent Recall the \hyperref[researchsubquestion:does_it_compare]{second research subquestion}:\vspace{0.5\baselineskip}\\
\textit{\quotes{Can the proposed method combine learning and planning for push en drive applications? Can the proposed method complete tasks, and how does it compare against the state of the art?}}\vspace{0.5\baselineskip}

The proposed hypothesis graph with the associated knowledge graph was able to perform as well or better compared to teh state of the art. The proposed method was tested against existing methods which only a subset of tasks, or 2 of the 3 topics. In comparison with with\cite{sabbaghnovin_model_2021} the proposed method was shown to perform similarly in prediction errors, and final displacement error. The proposed method is not designed to optimise a global plan, thus unnecessary driving and pushing can occur. 
\vspace{\baselineskip}

\todo[inline]{anser the main research question}
Now the main question can be answered...


\section{Drawbacks}
\cite{vega-brown_asymptotically_2020} optimises a global path (let's say shortest path to complete a task), the proposed method takes a random subtask which does not even try to converge toward a global optimal path. 

\section{Future work}%
\label{sec:future_work}
\todo[inline]{how could the closed-world assumption be removed to resample the real world more?}
\todo[inline]{how could the perfect object sensor assumption be removed to resample the real world more?}
\todo[inline]{how could the task as commutative assumption be removed to resample the real world more?}
\todo[inline]{how could the objects do not tip over assumption be removed to resample the real world more?}
\todo[inline]{The hgraph could think of new hypotheses during execution time if it would operate async, which would greatly improve it.}

