\subsection{Definition}
\label{subsec:hgraph_definition}

\todo[inline]{definition of state, node, }

\todo[inline]{introduce the NAMO problem, which itself is NP-hard}


\todo[inline]{introduce the citation \cite{ellis_navigation_2022}}





\textbf{Lifetime and types of edges}
\newline
An edge denotes that input commands are sent to the robot and as a result the robot can drive around changing it's state. Obstacles in a certain state represented by a node in the hgraph can be connected to other nodes via edges.
The edge contains all necessary components to sent input to the robot resulting in an obstacle reaching the target state residing in the node where the edge points toward. \vspace{\baselineskip}

There exist two types of edges, identification edges and action edges. An identification edge is responsible for sending an input sequence to the system and recording the system output which the simulation returns. An input/output sequence and assumptions on the system are the basis for system identification techniques discussed in \cref{subsec:sys_iden_and_control_methods}. The goal is to create a dynamical model which if augmented with an corresponding controller is closed-loop stable. \vspace{\baselineskip}

An action edge is responsible for putting an obstacle at a target state, the goal is to reach the target state, without triggering the fault detector and scoring well at internal metrics elaborated in \cref{subsec:edge_metrics}.

\todo[inline]{the lifetime of an identification edge}

The different status an edge can have is listed below. An edge status can only change according to the directed arrows in \cref{tikz: lifetime_action_edge}

\definecolor{lavenderIndigo}{RGB}{160, 110, 224}
\definecolor{batteryChargedBlue}{RGB}{22, 164, 216}
\definecolor{skyBlue}{RGB}{96, 219, 232}
\definecolor{kiwi}{RGB}{139, 211, 70}
\definecolor{minionYellow}{RGB}{239, 223, 72}
\definecolor{deepSaffron}{RGB}{249, 165, 44}
\definecolor{sinopia}{RGB}{214, 78, 18}

\begin{figure}[H]
\centering
\begin{tikzpicture}[node distance = 2cm, auto]
    \node [block, fill=lavenderIndigo] (init) {Initialised};
    \node [block, fill=batteryChargedBlue, below of=init] (path_exist) {Path Exists};
    \node [block, fill=skyBlue, below of=path_exist] (system_model) {System Model};
    \node [block, fill=kiwi, below of=system_model] (path_planned) {Path Planned};
    \node [block, fill=minionYellow, below of=path_planned] (executing) {Executing};
    \node [block, fill=deepSaffron, below of=executing] (completed) {Completed};
    \node [block, fill=sinopia] (failed) at ([xshift=4cm]$(system_model)!0.5!(path_planned)$) {Failed};
    
    % arrows
    \draw [-stealth] ([xshift=-2cm]init.west) to node[near start,above]{select controller} (init.west);
    \draw [-stealth] (init.south) to node[left]{graph-based path estimation} (path_exist.north);
    \draw [-stealth] (path_exist.west) [out=215,in=145] to node[left]{system identification} ([yshift=0.3cm] system_model.west);
    \draw [-stealth] ([yshift=-0.3cm] system_model.west) [out=215,in=145] to node[left]{motion planning} ([yshift=0.3cm] path_planned.west);
    \draw [-stealth] ([yshift=-0.3cm] path_planned.west) [out=215,in=145] to  node[left]{go to execution loop} ([yshift=0.3cm] executing.west);
    \draw [-stealth] (executing.south) to node[left]{evaluate with metrics} (completed.north);
    % edges to failed
    \draw [-stealth] (init.east) [out=0, in=90] to node[right]{path non-existence proven}  ([xshift=0.3cm]failed.north);
    \draw [-stealth] (path_exist.east) [out=0, in=90] to node[xshift=-0.4cm,yshift=0.55cm, above]{\shortstack[l]{system\\identification\\error}}  ([xshift=-0.3cm]failed.north);
    \draw [-stealth] (system_model.east) [out=0, in=180] to node[xshift=0.1cm, yshift=0.3cm, above]{\shortstack[l]{motion\\planning\\error}} ([yshift=0.3cm]failed.west);
    node[right]{motion planning error}  
    ([yshift=-0.3cm]failed.west);
    \draw [-stealth] (executing.east) [out=0, in=-90] to node[right]{fault detected}(failed.south);
\end{tikzpicture}
\caption{The status of an action edge and the most important indicator to change status}
\label{tikz: lifetime_action_edge}
\end{figure}

\begin{enumerate}
    \item[INITIALISED] The edge is created with a source and target node which are present in the hypothesis graph. A choice of controller is made.
    \item[PATH EXISTS] A graph-based search is performed to validate if the target state is reachable assuming that the system is holonomic.
    \item[SYSTEM MODEL] A dynamics system model is provided to the controller residing in the edge.
    \item[PATH PLANNED] Resulting from a sampble-based planner, a path from start to target state is provided. 
    \item[EXECUTING] The edge is currently sending input toward the robot. 
    \item[COMPLETED] The edge has driven the system toward it's target state and it's performance has been calculated.
    \item[FAILED] An error occurred, yielding the edge unusable. 
\end{enumerate}

\newpage
\newgeometry{left=1.1cm,bottom=0.1cm,top=1.9cm,headsep=0.1in,heightrounded}

\begin{figure}[H] 
\centering
\begin{tikzpicture}[node distance = 3cm]
    % Nodes
    \node [block, fill=yellow!50, line width=2pt, dashed] (first) {create start and target node};
    
    % legend
    \node[text width=2.8cm, yshift=1cm, right of=first, node distance=7cm, text centered, rounded corners, minimum height=1em, label={[name=lab, yshift=0.4cm, left]\textbf{Legend}}] (legend1) {\small Update KGraph};
    \node[rectangle, draw, left of=legend1, fill=green!50, rounded corners, minimum height=1em, minimum width=1cm, node distance=2cm] (legend1color) {};
    
    \node[text width=2.8cm, below of=legend1, text centered, minimum height=1em, node distance=0.7cm] (legend2) {\small Query KGraph};
    \node[rectangle, draw, left of=legend2, fill=red!40, rounded corners, minimum height=1em, minimum width=1cm, node distance=2cm] (legend2color) {};
   
    \node[text width=2.8cm, below of=legend2, text centered, minimum height=1em, node distance=0.7cm] (legend3) {\small Update C-Space};
\node[rectangle, draw, left of=legend3, fill=yellow!50, rounded corners, minimum height=1em, minimum width=1cm, node distance=2cm] (legend3color) {};
    
    \node[text width=2.8cm, below of=legend3, text centered, minimum height=1em, node distance=0.7cm] (legend4) {\small action in HGraph};
    \node[rectangle, draw, left of=legend4, rounded corners, minimum height=1em, minimum width=1cm, node distance=2cm, line width=2pt, dashed] (legend4color) {};
 
    \node[text width=2.8cm, below of=legend4, text centered, minimum height=1em, node distance=0.7cm] (legend5) {\small action in C-Space};
\node[rectangle, draw, left of=legend5, rounded corners, minimum height=1em, minimum width=1cm, node distance=2cm, line width=2pt] (legend5color) {};

    % nodes, Path exists 
    \node [decision, below of=first, node distance=3cm, line width=2pt] (path_existence) {Path non-existence};
    \node [decision, left of=path_existence, node distance=4.5cm, line width=2pt, dashed] (subtasks) {Is there a Subtask};
    \node [block, above of=subtasks, node distance=2.7cm] (no_solution_found) {no solution found};
    
    % nodes, Knowledge available
    \node [decision, fill=red!40, below of=path_existence, node distance=3cm, inner sep=0.5mm] (know_avail) { Knowledge Available };
    \node [decision, fill=red!40, right of=know_avail, node distance=3.5cm, inner sep=0.5mm] (know_good) {Knowledge Usable};
    \node [decision, right of=know_good, node distance=3.5cm, inner sep=0.5mm] (movable) {Obstacle Movable or Unkown};
    \node [block, left of=know_avail, node distance=3cm, line width=2pt, dashed] (impossible) {impossible task, abort subtask};
    
    % nodes, Generate new transition
    \node [decision, below of=know_avail, node distance=3.2cm, line width=2pt, inner sep=0.5mm, dashed] (goto_sys_iden) {Generate new Transition};
    \node[block, right of=goto_sys_iden, node distance=3.5cm, line width=2pt, dashed] (no_trans_found) {no transaction found, abort subtask};
    
    
    % Motion/Manipulation planning 
    \node [decision, below of=goto_sys_iden, node distance=3.5cm] (single_multi) {Single, Multi- body};
    \node [decision, line width=2pt, dashed, left of=single_multi, node distance=3.7cm] (model_avail_single) {Model available};
    \node [decision, line width=2pt, dashed, right of=single_multi, node distance=3.7cm] (model_avail_multi) {Model available};
    \node [block, line width=2pt, dashed, left of=model_avail_single, node distance=2.7cm] (sys_iden_single) {Add subtask Sys. Iden.};
    \node [block, line width=2pt, dashed, right of=model_avail_multi, node distance=3cm] (sys_iden_multi) {Add subtasks Go To + Sys. Iden.};
    \node [block, line width=2pt, dashed, below of=single_multi, node distance=3cm] (move_object) {Add subtask to Move Object};
    \node [block, line width=2pt, left of=move_object, node distance=3.7cm] (motion_planning) {Motion Planning};
    \node [block, line width=2pt, right of=move_object, minimum width=2.3cm, node distance=3.7cm] (manipulation_planning) {Manipulation Planning};
    
    % nodes, Path to target
    \node [decision, below of =move_object, node distance=3.5cm, line width=2pt, dashed] (global_path) {Path to Target}; 
    \node [decision, right of=global_path, node distance=3.5cm, line width=2pt, dashed] (first_action) {First Action Planned};
    \node [decision, right of=first_action, diagonal fill={yellow!50}{green!50}, node distance=3cm] (execute) {Execute};
     
    % nodes, Target node reached 
    \node [decision, below of=global_path, node distance=3cm, line width=2pt, dashed] (target_node_reached) {Target Node Reached};
    \node [block, left of=target_node_reached, node distance=3cm] (end) {Task successfully executed};
    
    % Edges
    \path[line] ++(0,1.5) -- node[left]{task} (first);
    \path[line] (first) -- node[midway](to_path_exists){}(path_existence); 
    
    % edges, Path exists 
    \path[line] (path_existence) -- node[midway, above, left] {no} (impossible.north east);
    \path[line] (subtasks.north) --  node[left] {no} (no_solution_found);
    \path[line] (path_existence) -- node[left] {yes} (know_avail); 
    \path[line] (subtasks.east) -- node[above] {yes} (path_existence.west);
    
    % edges, Knowledge available
    \path[line] (know_avail) -- node[above] {yes} (know_good); 
    \path[line] (know_avail) -- node[left](toward_new_trans) {no} (goto_sys_iden); 
    \draw[->] (know_good.east) -- node[above] {yes} (movable.west);
    
    % \draw[-]  ([xshift=3.2mm]toward_new_trans.center) -| node[near start, above] {no} (know_good.south);
    \draw[-](impossible.west) -- +(-0.47,0); 
     
    \draw[->]  ([xshift=1.75cm, yshift=7.7cm]know_avail.center) --  node[at start, above] {action suggestions} ([xshift=1.75cm, yshift=1.75cm]know_avail.center) -- (know_avail.north east);
    \draw[->]  ([xshift=1.75cm, yshift=1.75cm]know_avail.center) -- (know_good.north west);
    \draw [draw=white,double distance=\pgflinewidth,ultra thick] (path_existence.east) -- +(2cm,0);
    
    % edges, Generate new transition
    \draw[-] (move_object.south) |- +(-8.2,-0.3);
    \draw [draw=white,double=black,double distance=\pgflinewidth,ultra thick] (motion_planning.south) -- +(0,-1cm);
    \draw[-stealth] (motion_planning.south)  -- ([yshift=-1cm]motion_planning.south) -| node[near start, above] {success} (global_path.north);
    \draw[-] (manipulation_planning.south) -- ([yshift=-1cm]manipulation_planning.south) -| node[near start, above] {success} (global_path.north);
    \draw[-] (motion_planning.west) -- node[above] {failure} +(-3.47,0);
    \draw[-] (manipulation_planning.east) -| node[near start, above] {failure} ([xshift=4.7cm,yshift=-0.6cm]no_trans_found.south) -- ([yshift=-0.6cm]no_trans_found.south);
    
    % edges, Single/Multi body
    \draw[-stealth] (single_multi.west) -- node[above] {single} (model_avail_single);
    \draw[-stealth] (single_multi.east) -- node[above] {multi} (model_avail_multi);
    \draw[-stealth] (model_avail_single.south) -- node[left] {yes} (motion_planning.north);
    \draw[-stealth] (model_avail_single.west) -- node[above] {no} (sys_iden_single);
    \draw[-stealth] (model_avail_multi.south) -- node[near start, left] {yes} (manipulation_planning.north);
    \draw[-stealth] (model_avail_multi.east) -- node[above] {no} (sys_iden_multi);
    \draw[-stealth] (motion_planning.east) -- node[above] {blockade} (move_object);
    \draw[-stealth] (manipulation_planning.west) -- node[above] {blockade} (move_object);
    \draw[-stealth] (goto_sys_iden) -- node[above] {failure} (no_trans_found);
    \draw[-] (sys_iden_single.north) --  ([yshift=1.07cm]sys_iden_single.north);
    \draw[-] (sys_iden_multi.north) |-  ([yshift=-0.6cm]no_trans_found.south);
    \draw[-] (no_trans_found.south) -- ++(0,-0.6cm) --([xshift=-8cm, yshift=-0.6cm]no_trans_found.south);
    \draw [draw=white,double=black,double distance=\pgflinewidth,ultra thick] (goto_sys_iden.south) -- node[at start, left] {success}(single_multi.north);
    \draw[-stealth] ([yshift=-0.5cm]goto_sys_iden) -- (single_multi.north);
    
    \draw[-] (movable.south) |- node[near start, left] {yes} ([xshift=-1.5cm, yshift=-1.4cm]movable.south) |- ([yshift=0.2cm]single_multi.north);
    \draw [draw=white,double distance=\pgflinewidth,ultra thick]  (movable.north) |- ([xshift=-8cm, yshift=0.1cm]movable.north);
    \draw[-] (movable.north) |- node[above]{no}([xshift=-8.08cm, yshift=0.1cm]movable.north);
    % HERE
    \draw [draw=white,double=black,double distance=\pgflinewidth,ultra thick] ([xshift=5.5cm,yshift=0.3cm]single_multi.north) -- ([xshift=5.5cm, yshift=2cm]single_multi.north);
    % \draw[-] (know_good.east) -| node[above]{yes} ([xshift=5.5cm, yshift=0.2cm]single_multi.north) -- ([yshift=0.2cm]single_multi.north);
    
    % edges, Path to target
    \path[line] (global_path) -- node[above] {yes} (first_action);
    \path[line] (first_action.east) -- node[above] {yes} (execute);
    \path[line] (global_path.west) -| node[left, below, near start] {no} ([xshift=-7cm, yshift=8.31cm]global_path.west) -|  (subtasks.south); 
   
    \draw[-stealth] (first_action.north) |- node[near end, above] {no} ([xshift=1.7cm, yshift=0.15cm]first_action.north) |- ([yshift=-0.35cm]single_multi.south) -- (single_multi.south);
    \draw [draw=white,double=black,double distance=\pgflinewidth,ultra thick] (manipulation_planning.east) -- +(1cm,0);
    \draw [draw=white,double=black,double distance=\pgflinewidth,ultra thick] (manipulation_planning.north) -- +(0,1cm);
    
    \draw[-stealth] ([yshift=0.2cm, xshift=0.2cm]execute.south east) --  ([yshift=-0.8cm, xshift=1.2cm]execute.south east) -- node[at end, left] {robot input, action feedback} +(0,-2.7cm);
    
    \draw[stealth-] ([yshift=-0.2cm, xshift=-0.2cm]execute.south east) --  ([yshift=-1.2cm, xshift=0.8cm]execute.south east) -- node[left, at end] {sensor measurements} +(0, -1.8cm);
    
    \path[line] (execute.south) |- node[near start, left] {success} (target_node_reached.east);
    \draw[-stealth] (execute.east) -- node[above] {failure} ([xshift=1.5cm]execute.east) |- (path_existence.east);
    
    
    % edges, Target node reached 
    \path[line] (target_node_reached.north) -- node[left] {no} (global_path.south);
    \path[line] (target_node_reached.west) -- node[above] {yes} (end.east);

\end{tikzpicture}
\caption{Flowchart displaying the hypothesis graph's workflow.}
\label{figure:flowchart_hgraph} 
\end{figure}

\todo[inline]{update the figure above here}
\restoregeometry
