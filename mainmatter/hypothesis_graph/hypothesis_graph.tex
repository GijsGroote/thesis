\section{Hypothesis Graph}%
\label{sec:hgraph}
The \ac{hgraph} is responsible for generating action sequences, called hypothesis (consisting of a list of succesive edges from start to target node in the \ac{hgraph}) that complete a single subtasks. When all subtasks in a task are completed, the \ac{hgraph} halts and concludes the task succesfully completed. A search in the joint configuration space is avoided because an edge only operates in a single mode of dynamics, such as driving or pushing. When a object cannot directly be steered toward it's target location new nodes are generated which need to be completed before the original object can be steered toward it's target location. An example of when an object cannot directly be steered toward it's target state is because the path is blocked by another object. An hypothesis, consisting of a list of edges that represent actions in the robot environment might succeed or fail. \Cref{tikz:flowchart_hgraph} displays a flowchart explaining how new nodes and edges are generated in the \ac{hgraph}. Succesfully completed edges eventually result in completed subtasks, failed edges trigger replanning that will restart the search to a hypothesis.\bs

\Cref{subsec:hgraph_definition} defines the \ac{hgraph} and components, then~\cref{subsec:two_loops} eleborates how the \ac{hgraph} searches for a solution in the joint configuration space. Followed by subsections elaborating the methods used by the \ac{hgraph} such as system identification and control methods in \cref{subsec:sys_iden_and_control_methods}, path estimation in \cref{subsec:estimating_path_existence}, motion planning in \cref{subsec:motion_planning}, manipulation planning in \cref{subsec:manipulation_planning} and fault detection in \cref{subsec:fault_detection}.\bs

% For every subtask in a task, a start and a target node is created, the hypothesis graph tries to connect the starting node to the corresponding target node by adding node and edges. The choice and content of these nodes and edges is based on local planning and randomisation, elaborated in~\cref{subsec:estimating_path_existence,subsec:motion_planning,subsec:manipulation_planning}. A successive path from a starting node to the corresponding target node is called a hypothesis. During the search for a hypothesis, the hypothesis graph resides in the \textbf{search loop}, when traversing over edges toward the target node the hypothesis graph resides in the \textbf{execution loop}, both loops are elaborated in \cref{subsec:2_loops}. When the hypothesis graph traverses over an edge, fault detectors monitors the progress, elaborated in \cref{subsec:fault_detection}. Finally an example of an hypothesis graph is given in \cref{subsec:hgraph_example}.\bs
%
% After execution, the traversed edge enters a review period, during which performance is checked in various metrics. The review is stored in a database, called the \textbf{knowledge graph} which serves to collect edges and rank them based on performance. The knowledge graph is defined in \cref{subsec:kgraph_definition}, metrics to rank edges can be found in \cref{subsec:edge_metrics} and an example is displayed in \cref{subsec:kgraph_example}.
% \todo[inline]{come back to joint configuration space, and tell the hgraph cuts the different modes of dynamics with an edge for every mode of dynamics}


\subsection{Definition}
\label{subsec: hgraph_definition}

\todo[inline]{definition of state, node, }



\textbf{Lifetime and types of edges}
\newline
An edge denotes that input commands are sent to the robot and as a result the robot can drive around changing it's state. Obstacles in a certain state represented by a node in the hgraph can be connected to other nodes via edges.
The edge contains all necessary components to sent input to the robot resulting in an obstacle reaching the target state residing in the node where the edge points toward. \\

There exist two types of edges, identification edges and action edges. An identification edge is responsible for sending an input sequence to the system and recording the system output which the simulation returns. An input/output sequence and assumptions on the system are the basis for system identification techniques discussed in \cref{subsec: sys_iden_and_control_methods}. The goal is to create a dynamical model which if augmented with an corresponding controller is closed-loop stable. \\

An action edge is responsible for putting an obstacle at a target state, the goal is to reach the target state, without triggering the fault detector and scoring well at internal metrics elaborated in \cref{subsec: edge_metrics}.

\todo[inline]{the lifetime of an identification edge}

The different status an edge can have is listed below. An edge status can only change according to the directed arrows in \cref{tikz: lifetime_action_edge}

\definecolor{lavenderIndigo}{RGB}{160, 110, 224}
\definecolor{batteryChargedBlue}{RGB}{22, 164, 216}
\definecolor{skyBlue}{RGB}{96, 219, 232}
\definecolor{kiwi}{RGB}{139, 211, 70}
\definecolor{minionYellow}{RGB}{239, 223, 72}
\definecolor{deepSaffron}{RGB}{249, 165, 44}
\definecolor{sinopia}{RGB}{214, 78, 18}

\begin{figure}[H]
\centering
\begin{tikzpicture}[node distance = 2cm, auto]
    \node [block, fill=lavenderIndigo] (init) {Initialised};
    \node [block, fill=batteryChargedBlue, below of=init] (path_exist) {Path Exists};
    \node [block, fill=skyBlue, below of=path_exist] (system_model) {System Model};
    \node [block, fill=kiwi, below of=system_model] (path_planned) {Path Planned};
    \node [block, fill=minionYellow, below of=path_planned] (executing) {Executing};
    \node [block, fill=deepSaffron, below of=executing] (completed) {Completed};
    \node [block, fill=sinopia] (failed) at ([xshift=4cm]$(system_model)!0.5!(path_planned)$) {Failed};
    
    % arrows
    \draw [-stealth] ([xshift=-2cm]init.west) to node[near start,above]{select controller} (init.west);
    \draw [-stealth] (init.south) to node[left]{graph-based path estimation} (path_exist.north);
    \draw [-stealth] (path_exist.west) [out=215,in=145] to node[left]{system identification} ([yshift=0.3cm] system_model.west);
    \draw [-stealth] ([yshift=-0.3cm] system_model.west) [out=215,in=145] to node[left]{motion planning} ([yshift=0.3cm] path_planned.west);
    \draw [-stealth] ([yshift=-0.3cm] path_planned.west) [out=215,in=145] to  node[left]{go to execution loop} ([yshift=0.3cm] executing.west);
    \draw [-stealth] (executing.south) to node[left]{evaluate with metrics} (completed.north);
    % edges to failed
    \draw [-stealth] (init.east) [out=0, in=90] to node[right]{path non-existence proven}  ([xshift=0.3cm]failed.north);
    \draw [-stealth] (path_exist.east) [out=0, in=90] to node[xshift=-0.4cm,yshift=0.55cm, above]{\shortstack[l]{system\\identification\\error}}  ([xshift=-0.3cm]failed.north);
    \draw [-stealth] (system_model.east) [out=0, in=180] to node[xshift=0.1cm, yshift=0.3cm, above]{\shortstack[l]{motion\\planning\\error}} ([yshift=0.3cm]failed.west);
    node[right]{motion planning error}  
    ([yshift=-0.3cm]failed.west);
    \draw [-stealth] (executing.east) [out=0, in=-90] to node[right]{fault detected}(failed.south);
\end{tikzpicture}
\caption{The status of an action edge and the most important indicator to change status}
\label{tikz: lifetime_action_edge}
\end{figure}

\begin{enumerate}
    \item[INITIALISED] The edge is created with a source and target node which are present in the hypothesis graph. A choice of controller is made.
    \item[PATH EXISTS] A graph-based search is performed to validate if the target state is reachable assuming that the system is holonomic.
    \item[SYSTEM MODEL] A dynamics system model is provided to the controller residing in the edge.
    \item[PATH PLANNED] Resulting from a sampble-based planner, a path from start to target state is provided. 
    \item[EXECUTING] The edge is currently sending input toward the robot. 
    \item[COMPLETED] The edge has driven the system toward it's target state and it's performance has been calculated.
    \item[FAILED] An error occurred, yielding the edge unusable. 
\end{enumerate}
\subsection{The Search and the Execution loop}%
\label{subsec:two_loops}
\todo[inline]{introduce search and execution loop}
\todo[inline]{Adding nodes always happens, in a backward search fashion}

\newpage
\newgeometry{left=1.1cm,bottom=0.1cm,top=1.9cm,headsep=0.1in,heightrounded}

\begin{figure}[H] 
\centering
\begin{tikzpicture}[node distance = 3cm]
    % Nodes
    \node [block, fill=yellow!50, line width=2pt, dashed] (first) {create start and target nodes};
    
    % legend
    \node[text width=2.8cm, yshift=1cm, right of=first, node distance=7cm, text centered, rounded corners, minimum height=1em, label={[name=lab, yshift=0.4cm, left]\textbf{Legend}}] (legend1) {\small Update KGraph};
    \node[rectangle, draw, left of=legend1, fill=green!50, rounded corners, minimum height=1em, minimum width=1cm, node distance=2cm] (legend1color) {};
    
    \node[text width=2.8cm, below of=legend1, text centered, minimum height=1em, node distance=0.7cm] (legend2) {\small Query KGraph};
    \node[rectangle, draw, left of=legend2, fill=red!40, rounded corners, minimum height=1em, minimum width=1cm, node distance=2cm] (legend2color) {};
   
    \node[text width=2.8cm, below of=legend2, text centered, minimum height=1em, node distance=0.7cm] (legend3) {\small Update C-Space};
\node[rectangle, draw, left of=legend3, fill=yellow!50, rounded corners, minimum height=1em, minimum width=1cm, node distance=2cm] (legend3color) {};
    
    \node[text width=2.8cm, below of=legend3, text centered, minimum height=1em, node distance=0.7cm] (legend4) {\small action in HGraph};
    \node[rectangle, draw, left of=legend4, rounded corners, minimum height=1em, minimum width=1cm, node distance=2cm, line width=2pt, dashed] (legend4color) {};
 
    \node[text width=2.8cm, below of=legend4, text centered, minimum height=1em, node distance=0.7cm] (legend5) {\small action in C-Space};
\node[rectangle, draw, left of=legend5, rounded corners, minimum height=1em, minimum width=1cm, node distance=2cm, line width=2pt] (legend5color) {};

    % nodes, Path exists 
    \node [decision, below of=first, node distance=2.6cm, line width=2pt] (path_existence) {Estimate Path Existence};
    \node [decision, left of=path_existence, node distance=4.5cm, line width=2pt, dashed] (subtasks) {Is there an unfinished Subtask};
    \node [block, above of=subtasks, node distance=2.7cm] (no_solution_found) {no solution found};
    
    % nodes, Knowledge available
    \node [decision, fill=red!40, below of=path_existence, node distance=3.2cm, inner sep=0.5mm] (know_avail) { Knowledge Available };
    \node [decision, fill=red!40, right of=know_avail, node distance=3.5cm, inner sep=0.5mm] (know_good) {Knowledge Usable};
    \node [decision, right of=know_good, node distance=3.5cm, inner sep=0.5mm] (movable) {Object Movable or Unknown};
    \node [block, left of=know_avail, node distance=3cm, line width=2pt, dashed] (impossible) {impossible task, abort subtask};
    
    % nodes, Generate new edge
    \node [decision, below of=know_avail, node distance=3.2cm, line width=2pt, inner sep=0.5mm, dashed] (goto_sys_iden) {Generate Random Edge};
    \node[block, right of=goto_sys_iden, node distance=3.5cm, line width=2pt, dashed] (no_trans_found) {No more edges available, abort subtask};
    
    
    % Motion/Manipulation planning 
    \node [decision, below of=goto_sys_iden, node distance=3.5cm] (single_multi) {Driving or Pushing?};

    \node [decision, line width=2pt, dashed, left of=single_multi, node distance=3.7cm] (model_avail_single) {Model available};
    \node [decision, line width=2pt, dashed, left of=single_multi, node distance=3.7cm] (model_avail_single) {Model available};
    \node [decision, line width=2pt, dashed, right of=single_multi, node distance=3.7cm] (model_avail_multi) {Model available};
    \node [block, line width=2pt, dashed, left of=model_avail_single, node distance=2.7cm] (sys_iden_single) {Add Sys. Iden. Node};
    \node [block, line width=2pt, dashed, right of=model_avail_multi, node distance=3cm] (sys_iden_multi) {Add Sys. Iden. Node};
    \node [block, line width=2pt, dashed, below of=single_multi, node distance=3cm] (move_object) {Add Node to Move Object};
    \node [block, line width=2pt, left of=move_object, node distance=3.7cm] (motion_planning) {Motion Planning};
    \node [block, line width=2pt, right of=move_object, minimum width=2.3cm, node distance=3.7cm] (manipulation_planning) {Manipulation Planning};
    \node [block, line width=2pt, dashed, minimum width=2.3cm, below of=move_object] (drive_to_object) {Add drive subtask to object};

    % nodes, Path to target
    \node [decision, below right of=drive_to_object, node distance=4.0cm, line width=2pt, dashed] (first_action) {First Action Planned};
    \node [decision, below left of=drive_to_object, node distance=3.5cm, line width=2pt, dashed] (global_path) {Path to Target}; 

    % \node [block, line width=2pt, dashed, minimum width=2.3cm] (drive_to_object) at ([xshift=0.1cm]$(move_object)!0.5!(global_path)$) {Add drive subtask to object};
    \node [decision, right of=first_action, diagonal fill={yellow!50}{green!50}, node distance=3cm] (execute) {Execute};
     
    % nodes, Target node reached 
    \node [decision, below of=global_path, node distance=3cm, line width=2pt, dashed] (target_node_reached) {Target Node Reached};
    \node [block, left of=target_node_reached, node distance=3cm] (end) {Task successfully executed};
    
    % Edges
    \path[line] ++(0,1.5) -- node[left]{task} (first);
    \path[line] (first) -- node[midway](to_path_exists){}(path_existence); 
    
    % edges, Path exists 
    \path[line] (path_existence) -- node[midway, above, left] {No path found} (impossible.north east);
    \path[line] (subtasks.north) --  node[left] {no} (no_solution_found);
    \path[line] (path_existence) -- node[xshift=0.08cm, yshift=0.35cm, right] {path found} (know_avail); 
    \path[line] (subtasks.east) -- node[above] {yes} (path_existence.west);
    
    % edges, Knowledge available
    \path[line] (know_avail) -- node[above] {yes} (know_good); 
    \path[line] (know_good) -- node[yshift=0.1cm, above] {no} (goto_sys_iden); 
    \path[line] (know_avail) -- node[left](toward_new_trans) {no} (goto_sys_iden); 
    \draw[->] (know_good.east) -- node[above] {yes} (movable.west);
    
    % \draw[-]  ([xshift=3.2mm]toward_new_trans.center) -| node[near start, above] {no} (know_good.south);
    \draw[-](impossible.west) -- +(-0.47,0); 
     
    \draw[->]  ([xshift=1.75cm, yshift=7.3cm]know_avail.center) --  node[at start, above] {action suggestions} ([xshift=1.75cm, yshift=1.75cm]know_avail.center) -- (know_avail.north east);
    \draw[->]  ([xshift=1.75cm, yshift=1.75cm]know_avail.center) -- (know_good.north west);
    \draw [draw=white,double distance=\pgflinewidth,ultra thick] (path_existence.east) -- +(2cm,0);
    
    % edges, Generate new edge
    \draw[-] (move_object.south) |- +(-8.2,-0.3);
    \draw [draw=white,double=black,double distance=\pgflinewidth,ultra thick] (motion_planning.south) -- +(0,-1cm);
    \draw[-stealth] (motion_planning.south)  -- ([yshift=-1cm]motion_planning.south) -| node[near start, above] {success} (global_path.north);
    \draw[-stealth] (manipulation_planning.south) |- node[near start, right] {success} (drive_to_object.east);
    \draw[-] (drive_to_object.west) -| (global_path.north);
    \draw[-] (motion_planning.west) -- node[above] {failure} +(-3.47,0);
    \draw[-] (manipulation_planning.east) -| node[near start, above] {failure} ([xshift=4.7cm,yshift=-0.6cm]no_trans_found.south) -- ([yshift=-0.6cm]no_trans_found.south);
    
    % edges, Single/Multi body
    \draw[-stealth] (single_multi.west) -- node[above] {driving} (model_avail_single);
    \draw[-stealth] (single_multi.east) -- node[above] {pushing} (model_avail_multi);
    \draw[-stealth] (model_avail_single.south) -- node[left] {yes} (motion_planning.north);
    \draw[-stealth] (model_avail_single.west) -- node[above] {no} (sys_iden_single);
    \draw[-stealth] (model_avail_multi.south) -- node[near start, left] {yes} (manipulation_planning.north);
    \draw[-stealth] (model_avail_multi.east) -- node[above] {no} (sys_iden_multi);
    \draw[-stealth] (motion_planning.east) -- node[above] {blockade} (move_object);
    \draw[-stealth] (manipulation_planning.west) -- node[above] {blockade} (move_object);
    \draw[-stealth] (goto_sys_iden) -- node[above] {failure} (no_trans_found);
    \draw[-] (sys_iden_single.north) --  ([yshift=1.07cm]sys_iden_single.north);
    \draw[-] (sys_iden_multi.north) |-  ([yshift=-0.6cm]no_trans_found.south);
    \draw[-] (no_trans_found.south) -- ++(0,-0.6cm) --([xshift=-8cm, yshift=-0.6cm]no_trans_found.south);
    \draw [draw=white,double=black,double distance=\pgflinewidth,ultra thick] (goto_sys_iden.south) -- node[at start, left] {success}(single_multi.north);
    \draw[-stealth] ([yshift=-0.5cm]goto_sys_iden) -- (single_multi.north);
    
    \draw[-] (movable.south) |- node[near start, left] {yes} ([xshift=-1.5cm, yshift=-1.4cm]movable.south) |- ([yshift=0.2cm]single_multi.north);
    \draw [draw=white,double distance=\pgflinewidth,ultra thick]  ([xshift=-1cm]movable.north) -- ([xshift=-8cm]movable.north);

    \draw[-] (movable.north) -- node[above]{no}([xshift=-8.23cm]movable.north);
    % HERE
    \draw [draw=white,double=black,double distance=\pgflinewidth,ultra thick] ([xshift=5.5cm,yshift=0.3cm]single_multi.north) -- ([xshift=5.5cm, yshift=2cm]single_multi.north);
    % \draw[-] (know_good.east) -| node[above]{yes} ([xshift=5.5cm, yshift=0.2cm]single_multi.north) -- ([yshift=0.2cm]single_multi.north);
    
    % edges, Path to target
    \path[line] (global_path) -- node[above] {yes} (first_action);
    \path[line] (first_action.east) -- node[above] {yes} (execute);
    \path[line] (global_path.west) -| node[left, below, near start] {no} ([xshift=-3cm, yshift=8.31cm]global_path.west) -|  (subtasks.south); 
   
    \draw[-stealth] (first_action.north east) -- node[near end, right] {no} ([xshift=1.7cm, yshift=0.39cm]first_action.north) |- ([yshift=-0.35cm]single_multi.south) -- (single_multi.south);
    \draw [draw=white,double=black,double distance=\pgflinewidth,ultra thick] (manipulation_planning.east) -- +(1cm,0);
    \draw [draw=white,double=black,double distance=\pgflinewidth,ultra thick] (manipulation_planning.north) -- +(0,1cm);
    
    \draw[-stealth] ([yshift=0.2cm, xshift=0.2cm]execute.south east) --  ([yshift=-0.8cm, xshift=1.2cm]execute.south east) -- node[at end, left] {robot input, action feedback} +(0,-2.7cm);
    
    \draw[stealth-] ([yshift=-0.2cm, xshift=-0.2cm]execute.south east) --  ([yshift=-1.2cm, xshift=0.8cm]execute.south east) -- node[left, at end] {sensor measurements} +(0, -1.8cm);
    
    \path[line] (execute.south) |- node[near start, left] {success} (target_node_reached.east);
    \draw[-stealth] (execute.east) -- node[above] {failure} ([xshift=1.5cm]execute.east) |- (path_existence.east);
    
    
    % edges, Target node reached 
    \path[line] (target_node_reached.north) -- node[left] {no} (global_path.south);
    \path[line] (target_node_reached.west) -- node[above] {yes} (end.east);
\end{tikzpicture}
\caption{Flowchart displaying the hypothesis graph's workflow.}
\label{tikz:flowchart_hgraph}% 
\end{figure}

\restoregeometry

\begin{figure}[H]
    \centering
    \includegraphics[width=5cm]{figures/boxer_robot.png}
    \caption{Pushing task through blocked corridor with the point robot, a green cube to push toward the target ghost state and a red blockade.}
    \label{fig:blocked_path_example_environment}
\end{figure}

\begin{figure}[H]
    \centering
    \begin{subfigure}{.5\textwidth}
    \centering
    \includegraphics[width=0.8\textwidth]{figures/boxer_robot.png}
    \caption{todo}
    \label{subfig:todo}
    \end{subfigure}%

    \begin{subfigure}{.5\textwidth}
    \centering
    \includegraphics[width=0.8\textwidth]{figures/boxer_robot.png}
    \caption{todo}
    \label{subfig:B}
    \end{subfigure}
    \caption{todo}
    \label{subfig:blocked_path_hgraph_exmple}
\end{figure}

\todo[inline]{update the figure above here, Martijn did not like single/multi body, completely replace these terms.}
\todo[inline]{make the colors different, some which can be visualised with the airlab monitors.}

\todo[inline]{here are some example hgraph's required}

\todo[inline]{should I do an example Hgraph here? that requires target ghost positions. yes implement a }

As can be seen in \cref{tikz:flowchart_hgraph} there are some methods used which are still unexplained. Such as path non-existence, or motion planning. 
\todo[inline]{walk through the flowchart, what is actually happening here, It might be in your mind, but can the reader understand the flowchart without any additional context? clarify how edges are initialised, hypotheses are formed and how replanning occurs}


\subsection{Example}%
\label{subsec:hgraph_example}

\todo[inline]{here are some example hgraph's required}
\begin{figure}[H]
    \centering
    \includegraphics[width=5cm]{figures/boxer_robot.png}
    \caption{Pushing task through blocked corridor with the point robot, a green cube to push toward the target ghost state and a red blockade.}%
\label{fig:blocked_path_example_environment}
\end{figure}

\begin{figure}[H]
    \centering
    \begin{subfigure}{.5\textwidth}
    \centering
    \includegraphics[width=0.8\textwidth]{figures/boxer_robot.png}
    \caption{todo}%
    \label{subfig:todo}
    \end{subfigure}%

    \begin{subfigure}{.5\textwidth}
    \centering
    \includegraphics[width=0.8\textwidth]{figures/boxer_robot.png}
    \caption{todo}%
    \label{subfig:B}
    \end{subfigure}
    \caption{todo}%
    \label{subfig:blocked_path_hgraph_exmple}
\end{figure}



