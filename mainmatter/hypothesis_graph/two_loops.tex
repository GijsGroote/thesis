\subsection{The Search and the Execution loop}%
\label{subsec:two_loops}
Now that \ac{hgraph} is defined let's find how it reacts it reacts to various behavirour. A flowchart of the \ac{hgraph} is provided in \cref{tikz:flowchart_hgraph}, where two loops can be identified. The search loop and the execution loop, to forestall that the loops are not correctly identified in \cref{tikz:flowchart_hgraph}, see \cref{fig:two_loops_identified}

\begin{figure}[H]
    \centering
    \includegraphics[width=5cm]{figures/two_loops_identified}
    \caption{The search and execution loop.}
    \label{fig:two_loops_identified}
\end{figure}

Upon initialisation of the \ac{hgraph}

% list the properties of the hgraph and why 

\input{mainmatter/hypothesis_graph/hgraph_tikz_figure}

% list what happens in every block in the tikz figure above 

\todo[inline]{walk through the flowchart, what is actually happening here, It might be in your mind, but can the reader understand the flowchart without any additional context? clarify how edges are initialised, hypotheses are formed and how replanning occurs}

As can be seen in \cref{tikz:flowchart_hgraph} there are some methods used which are still unexplained. Such as path non-existence, or motion planning. 
