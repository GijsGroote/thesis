
\section{Backward Search}%
\label{sec:backward_search}


% What's Purpose: backward search / backward induction
\paragraph{Backward Search}
This thesis proposes a method where the robot should learn robot and object dynamics by interaction, and perform motion and manipulation planning whilst facing a wide variety of objects, tasks and environments. The 3 topics (learning system models, \ac{NAMO}, nonprehensile push manipulation) are bundled together with a technique known as \textbf{backward induction} also known as \textbf{backward tracing} or \textbf{backward search} (this thesis will refer to this technique with the term backward search).\todo[inline]{Marijn: strange sentence} The proposed method in this thesis and \citeauthor{sabbaghnovin_model_2021} proposed method solve comparable problems using different techniques, allowing for easy comparison between the results of \citeauthor{sabbaghnovin_model_2021} and the results from the proposed method in this thesis.\bs
\todo[inline]{GIjS: bundeled together?}
\todo[inline]{Martijn : explain backward induction, backward tracing and provide references}

\todo[inline]{this section, }
\cite{krontiris_dealing_2015}



% The planning problem arises when a path from the start to the target configuration for an object must be found. Such a path cannot cross through obstacle space and must be feasible for the system to track. Motion and manipulation planning are split up because there are some major differences. Motion planning only plans for the robot whilst manipulation planning plans for the robot and some objects. Manipulation planning bears more constraints than motion planning because constraints for the robot, the obstacle and constraints between the robot and the obstacle must be respected. A system model discussed in \cref{subsec:sys_iden} acts as a local planner and validates if a path is feasible. 


