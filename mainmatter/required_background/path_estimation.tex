\section{Estimating Path Existence}%
\label{sec:estimating_path_existence}
Trying to accomplish an impossible task is a waste of time and resources. For that reason, a path estimation algorithm estimates the existence of a path before a planner (discussed in \cref{sec:planning}) starts a dedicated search. The path estimation algorithm chosen consists of a graph-based search on a discretised version of the configuration space and can be described as follows.\bs

\quotes{The main idea is to discretise the configuration space with a finite discretisation. The emerged cells act as nodes in the graph, cells are connected through edges to nearby cells. Graph-based planners start from the cell containing the starting pose and search for the cell containing the target pose whilst avoiding cells which lie in obstacle space.}\bs

\todo[inline]{you still have to define the Dijkstra path estimation algorithm}
\todo[inline]{use the words "occupancy map" and "Dijkstra" in the definition, and \cite{zhang_simple_2008}}
\todo[inline]{visualise the path estimation algorithm}

\paragraph{Unfeasible solutions and an undecidable problem}
The path estimation algorithms does not take system constraints into account. It is thus possible that the path estimation algorithm finds a path from the start to the target configuration, but in reality, the path is unfeasible. An example is driving the boxer robot through a narrow, sharp corner. Whilst theoretically the robot would fit through the corner, the nonholonomic property of the robot prevents it from steering through such a thight corner. It is for the motion or manipulation planner to detect that the path is unfeasible.\\The path estimator suffers from another drawback than potentially providing unfeasible solutions, finding proof that there exists a path that is undecidable. Due to the chosen cell size during discretising the configuration space and the alignment of cells. An example is a quoridor that has exactly the width of the robot. The robot would exactly fit through such a quoridor, but detecting this narrow quoridor requires a number of connected cells that lie exactly in the centre line of the quoridor. Path non-existence on the other hand can be proven easily, because effects such as caging can be detected~\cite{chen_fast_2018}.\bs

Compared to planners which will be discussed in the next section, estimating the existence of a path is fast. Potentially is can even speed up planning because the estimated path can be converted to samples for the motion/manipulation planner, a set of initial samples is referred to as a \textit{warm start}. But let's not get ahead, \cref{sec:planning} is dedicated to planners.\bs

Thus in exceptional cases the path estimation algorithm could yield unfeasible paths and could fails to detect existing paths. Generally, it is used to detect if there exists no path from start to target, and to provide a \quotes{warm start} to the planner. \bs
