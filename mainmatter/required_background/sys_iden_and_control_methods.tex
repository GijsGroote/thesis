% \section{System Identification and Control Methods}%
% \label{sec:sys_iden_and_control_methods}
\todo[inline]{Gijs: what could you do with this section here}
Driving a system toward a desired state is performed by a controller. A system model is required during control design to guarantee stable closed-loop performance. First system models and system identification methods are discussed in the upcoming subsection, and then control methods are discussed.\bs
\todo[inline]{martijn: I don't think that prediction error is typically seen as a control aspect.}

\section{System Identification}%
\label{sec:sys_iden}
Understanding the world is captured by models, \textit{system models} that estimate the behaviour of real-world systems. For example what trajectory does an object take after receiving a push? The trajectory is dependent on the properties of the object (e.g. mass, geomtry), and interaction with its surroundings (e.g sliding friction). These properties and interactions are captured by a system model. Just as applicable constraints that are captured by a system model, see both robots in \cref{fig:example_robots}, the holonomic point robot in \cref{subfig:example_point_robot} can drive without constraints, and the boxer robot in \cref{subfig:example_boxer_robot} can drive forward, backwards and can rotate. A system model for robot driving for the boxer robot should thus capture that it is not able to drive directly sideways.\bs
\todo[inline]{Martijn: reader does nto know about boxer robot}

This thesis encounters only 2 different robots but many different objects. Since every object can be different in type, weight and dimensions. System identification methods are required to capture the variety of objects that the robot can encounter.\bs

\todo[inline]{Visualise test push sequence to collect IO data}
\todo[inline]{document your \ac{PEM} method}
\todo[inline]{document your \ac{IPEM} method}

\subsection{Control Methods}%
\label{subsec:control_methods}
Understanding the world is done by a system model, but to perform actions and manipulate objects control is required. A first requirement for a controller is that it should yield a stable closed-loop control, and then additional requirements are desired such as a low prediction error and low final placement error.

\paragraph{\acs{MPC} Control}
The \ac{MPC} controller can be tuned in many ways, for example, to steer the system to the goal aggressively or calmly by adjusting the maximum input and by putting a penalty on rapid input changes. The weights inside its internal objective function can be adjusted during execution time allowing to shift the controller focus from rapid movement to avoiding obstacle regions or converging to a target setpoint. The best feasible input is found every time step by optimising the objective function whilst respecting constraints. Solving an optimisation function to find the best feasible input generally yields robust control. It is however required that the system model is \ac{LTI}. System models for driving the robot can be estimated with \ac{LTI} models without compromising on model accuracy.
\todo[inline]{Document the \ac{MPC} method}

\paragraph{\acs{MPPI} Control}
A main advantage \ac{MPPI} has over \ac{MPC} control is that it is compatible with nonlinear system models. Whilst drive applications can be accurately estimated by linear models, push applications are harder to estimate with a linear model. Thus \ac{MPPI} is selected mainly for push applications.
\todo[inline]{Document the \ac{MPPI} method}

\begin{figure}[H]
\begin{minipage}{0.5\linewidth}
\begin{table}[H]
\centering
\begin{tabular}[t]{l | c | c}
  control sys. iden. & \ac{PEM} & \ac{IPEM} \\
  \hline
  \ac{MPC} & \cmark & \cmark \\
  \ac{MPPI} & \cmark & \cmark \\
\end{tabular}
\caption{Compatibility between control and system identification methods for drive actions.}%
\label{table:compatible_modules_drive}
\end{table}
\end{minipage}
\begin{minipage}{0.5\linewidth}
\begin{table}[H]
\centering
\begin{tabular}[t]{l | c | c }
  control sys. iden. & \ac{PEM} & \ac{LSTM}\\\hline
  \ac{MPC} & \cmark & \xmark\\
  \ac{MPPI} & \cmark & \cmark\\
\end{tabular}
\caption{Compatibility between control and system identification methods for push actions.}%
\label{table:compatible_modules_push}
\end{table}
\end{minipage}
\caption{Compatible : todo thos stuff TODOD}
\label{table:compatible_modules}
\end{figure}

\todo[inline]{Augment above with all implemented sys identification methods}
