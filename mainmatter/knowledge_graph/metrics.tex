\subsection{Edge Metrics}%
\label{subsec:edge_metrics}
The \ac{kgraph} keeps an ordered list of `good' and `bad' edge arguments (controller and system model). `Good' and `bad' are defined by edge metrics, these metrics are created after the completion of an edge, regardless of whether the edge was successfully completed or failed. An indication is given on why certain metrics matter in \Cref{table:review_edge_metrics}.

\noindent
\begin{table}[H]
\centering
\begin{tabular}%
{>{\raggedright\arraybackslash}p{0.25\textwidth}%
>{\raggedright\arraybackslash}p{0.65\textwidth}}
\acf{PE}&  To better compare prediction errors the \ac{PE} is summarized and average \ac{PE}. The average \ac{PE} is an indicator of an accurate system model but can give misleading results since \ac{PE} is also an indicator of unexpected collisions. Prediction error should thus only be used if there are no collisions detected. The average \ac{PE} comes with more flaws since the average is mostly determined by outliers, some unfortunate outliers in the \ac{PE} might for the largest part determine the average \ac{PE}. The average \ac{PE} will thus not be used because it is not robust enough.\\
\acf{TE}& For a low \ac{TE} the system model must be close to the real motion equations to yield a feasible path, the controller must be well tuned to be able to track that path and the controller and system model must be in collaboration, because the controller uses the system model to calculate system input. A low \ac{TE} tells multiple things, whilst a high \ac{TE} would indicate improvements could be gained in the controller, the system model or their collaboration.\\
ratio num\_succesfully completed edges and num\_total edges & Over time the \ac{kgraph} can recommend the same edge arguments multiple times. Logging the ratio of succeeding edges vs total edges builds an evident portfolio. Still, this metric has to be taken with a grain of salt because edges with equal edge arguments perform similar actions e.g.~pushing an object through a wide corridor is compared to pushing the same object through a narrow corridor. One could say \quotes{comparing apples with pears}.\\
the final position and \newline displacement error & The quality of the result is measured in the final position and displacement error. The importance should thus be stressed when ordering edge arguments.\\
planning time& With system identification, path estimation, motion or manipulation planning the planning time can vary in orders of magnitude between simple or more complex approaches. Planning time mainly serves to rank the slowest planners low, whilst not influencing the rank of fast and average planners.\\
run time& Also known as execution time would be a quality indicator if start and target states would be equal. Edges are recommended to solve similar tasks, where the path length between the start and target state is different. Thus planning time is not of any use to rank edges.\\
completion time = \newline run time + planning time & With the same argumentation as run time, completion time is not of any use to rank edges.\\
\end{tabular}
\caption{Edge metrics used to rank control methods from `good' to `bad'}
\label{table:review_edge_metrics}
\end{table}
