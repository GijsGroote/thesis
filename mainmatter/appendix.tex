\chapter{Appendix}%
\label{chap:appendix}

\appendix

\textit{This intro to chapter appendix.\bs}

% \section{System Identification Methods}
% \subsection{Prediction Error Minimisation}
% \subsection{Prediction Error Minimisation}
%
% \section{Control Methods}
% \subsection{\acs{MPC}}
% \subsection{\acs{MPPI}}

\chapter{Complexity Classes}%
\label{paragraph:complexity_classes}
Problems in class P have a solution which can be found in polynomial time, problems in \ac{NP} are problems for which a solution cannot found in polynomial time. For problems in \ac{NP}, when provided with a solution, verifying that the solution is indeed a valid solution can be done in polynomial time. \ac{NP-hard} problems are a class of problems which are at least as hard as the hardest problems in \ac{NP}. Problems that are \ac{NP-hard} do not have to be elements of NP. They may not even be decidable~\cite{pokharel_computational_2020}. This thesis or other recent studies in the references do not attempt to find an optimal solution. Instead, they provide a solution whilst guaranteeing properties such as near-optimality or probabilistic completeness. We conclude that the \ac{NAMO} problem combined with relocating objects to target positions fall in the category of \ac{NP-hard} problems because it can be reduced to the piano's mover problem.\bs

