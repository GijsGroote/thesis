\chapter*{Abstract}
\addcontentsline{toc}{chapter}{Abstract}
In the field of robotics, consider the following problem scenario: In a robot environment a robot must push objects to reference places while figuring out which objects can be pushed, what the best manipulation strategy is, or which objects are static, and cannot be pushed. The problem scenarion can be decomposed in three research topics which individually have received much attention from the research community; \textit{learning object dynamics}~\cite{cong_selfadapting_2020,seegmiller_vehicle_2013}, \textit{\acf{NAMO}}~\cite{chen_fast_2018,elbanhawi_samplingbased_2014,kingston_samplingbased_2018,lavalle_planning_2006} and \textit{nonprehensile pushing}~\cite{arruda_uncertainty_2017,bauza_dataefficient_2018,mericli_pushmanipulation_2015,stuber_featurebased_2018,stuber_let_2020,toussaint_sequenceofconstraints_2022}. However, because scientific papers that include all three topics are scarce, the combination of these research topics into one robot framework has not been explored sufficiently to leverage environmental knowledge to there fullest potential during action planning. A combination of the topics leads to an improvement in planning, faster execution times and rapidly concluding that rearangement tasks are unfeasible for a robot.\bs

To solve the problem presented in the first paragraph, this thesis proposes a robot framework that combines these three research topics. This framework comprises of three key components: the \textbf{\acl{halgorithm}}, the \textbf{\acl{hgraph}}, and the \textbf{\acl{kgraph}}. The \acl{halgorithm} is used to draw a hypothesis on how to relocate an object to a new pose by computing possible action sequences given certain robot skills. In doing so, the \acl{halgorithm} creates an \acl{hgraph} that encapsulates the structure of the action sequences and ensures the robot eventually halts. Once an hypothesis is carried out on the robot, information about the execution, such as the outcome, the type of controller used and other metrics, are stored in the \acl{kgraph}. The knowledge graph is populated over time, allowing the robot to learn, for instance, object properties and then refine the hypothesis to increase task performance, such as success rate and execution time.\bs

A new planning algorithm is proposed that can detect a blocked path, the \acl{halgorithm} relies on the newly proposed planner to generate action sequences and to free blocked paths. This planner extends the double tree \acl{RRT*} algorithm~\cite{chen_fast_2018}. The planners both construct a configuration space for an object and are provided with starting and target pose for that object. The planners then convert these poses to points in configuration space and searches for a path connecting the starting point to the target point. A key difference between the newly proposed planner and the existing planner lies in the ability to detect blocked paths. For the new planner, objects are initially classified as \quotes{unknown} and can later be categorized as either \quotes{movable} or \quotes{obstacle}. The object type information is then used when constructing the configuration space for the newly proposed planning algorithm. Its configuration space consists of the conventional free, obstacle, and unknown- and movable space.\bs

To carry out the investigation, a mobile robot in a robot environment is created with movable and unmovable objects. The robot is given a task that involves relocating a subset of the objects in the robot environment through driving and nonprehensile pushing. The task can be broken down into individual subtasks that consist of an object and a target pose. Planning for a push or drive action occurs with the newly proposed planning algorithm that, if successful, provides a path that can be tracked.\bs

The three topics can be combined into a robot framework because results indicate that task execution improves as the robot gains more experience in its environment. The proposed framework, which covers all three research topics, performs equivalent or better compared to the state-of-the-art frameworks that are specialized in only two out of three research topics~\cite{ellis_navigation_2022,sabbaghnovin_model_2021,scholz_navigation_2016,vega-brown_asymptotically_2020,wang_affordancebased_2020}.\bs

\begin{flushright}
{\makeatletter\itshape
    \@author\\
    Delft, \monthname{} \the\year{}
\makeatother}
\end{flushright}
