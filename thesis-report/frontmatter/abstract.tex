\chapter*{Abstract}
\addcontentsline{toc}{chapter}{Abstract}
In the field of robotics, much attention has been given to the research topics \textit{learning object dynamics}~\cite{cong_selfadapting_2020,seegmiller_vehicle_2013}
, \textit{\acf{NAMO}}~\cite{chen_fast_2018,elbanhawi_samplingbased_2014,kingston_samplingbased_2018,lavalle_planning_2006} and \textit{nonprehensile pushing}~\cite{arruda_uncertainty_2017,bauza_dataefficient_2018,mericli_pushmanipulation_2015,stuber_featurebased_2018,stuber_let_2020,toussaint_sequenceofconstraints_2022}. However, the combination of these three research topics into one robot framework has not been explored sufficiently. Scientific papers that include all three topics are scarce. This thesis proposes a robot framework that combines these three research topics to investigate if this combination improves task execution compared to a combination of two out of three topics. To carry out the investigation, a robot environment with movable and unmovable objects is created, and the robot is given a task to fulfill. The task involves relocating a subset of the objects in the robot environment and can be broken down into individual subtasks that consist of an object and target configuration. Finding an optimal action sequence to complete a task requires a search in the joint configuration space, which emerges when the configuration space of the robot is augmented with every configuration space of every movable object in the environment.\bs

When entering a new environment, objects are initially classified as \quotes{unknown} and can later be categorized as either \quotes{movable} or \quotes{obstacle}. The object type information is used in a new planning algorithm, which includes searching for both unknown and movable spaces, in addition to the conventional free and obstacle spaces. This framework comprises of three key components: the \textbf{\acl{halgorithm}}, the \textbf{\acl{hgraph}}, and the \textbf{\acl{kgraph}}. The \acl{halgorithm} explores the joint configuration space and generates a \acl{hgraph}, which produces non-deterministic action sequences known as \textit{hypotheses}. Since a hypothesis is an action sequence, it can be executed and complete a subtask. If a fault is detected during the execution of a hypothesis, the \acl{halgorithm} will stop the execution and look for a new hypothesis. The robot framework has the ability to learn the type of objects in the environment and, if they are movable, learn the best way to interact with them. The environmental information learned by the robot is stored in the proposed \acl{kgraph}. This graph stores feedback on actions in the form of a controller, system model, and a reviewing method. Using this reviewing method, the \acl{kgraph} can suggest actions to the \acl{halgorithm}.\bs

It can be concluded that the three topics can be combined because results indicate that task execution improves as the robot gains more experience in the environment it operates. The proposed framework, which covers all three research topics, performs comparably to specialized state-of-the-art frameworks that only cover two out of three topics~\cite{ellis_navigation_2022,sabbaghnovin_model_2021,scholz_navigation_2016,vega-brown_asymptotically_2020,wang_affordancebased_2020}.\bs

\begin{flushright}
{\makeatletter\itshape
    \@author\\
    Delft, \monthname{} \the\year{}
\makeatother}
\end{flushright}
