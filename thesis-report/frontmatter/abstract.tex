\chapter*{Abstract}
\addcontentsline{toc}{chapter}{Abstract}
In the field of robotics, consider the following problem scenario: In a robot environment a simple robot must push objects to reference places while figuring out which objects can be pushed, what the best manipulation strategy is, or which objects are static, and cannot be pushed. The problem scenario can be decomposed in three research topics which individually have received much attention from the research community; \textit{learning object dynamics}~\cite{cong_selfadapting_2020,seegmiller_vehicle_2013}, \textit{\acf{NAMO}}~\cite{chen_fast_2018,elbanhawi_samplingbased_2014,kingston_samplingbased_2018,lavalle_planning_2006} and \textit{nonprehensile pushing}~\cite{arruda_uncertainty_2017,bauza_dataefficient_2018,mericli_pushmanipulation_2015,stuber_featurebased_2018,stuber_let_2020,toussaint_sequenceofconstraints_2022}. A combination of these three topics could lead to improvements in planning, execution time, and reasoning, but it has not been explored in the literature.\bs

This thesis proposes a robot framework that combines these three research topics. This framework comprises of three key components: the \textbf{\acl{h-algorithm}}, the \textbf{\acl{h-graph}}, and the \textbf{\acl{k-graph}}. The \acl{h-algorithm} computes an hypothesis on how to relocate an object to a new pose by computing possible action sequences given certain robot skills. In doing so, the \acl{h-algorithm} creates an \acl{h-graph} that encapsulates the structure of the action sequences and ensures the robot eventually halts. Once an hypothesis is carried out on the robot, information about the execution, such as the outcome, the prediction error the type of controller used and other metrics, are stored in the \acl{k-graph}. The knowledge graph is populated over time, allowing the robot to learn, for instance, object properties and then refine the hypothesis computed to increase task performance, such as success rate and execution time.\bs

A new planning algorithm is proposed that can detect a when a path is blocked by an object, the \acl{h-algorithm} relies on the newly proposed planner to generate action sequences and to free blocked paths. This planner extends the double tree \acl{RRT*} algorithm~\cite{chen_fast_2018}. The planner constructs a configuration space for an object and is provided with starting- and target pose for that object. The planner then convert these poses to points in configuration space to then search for a path connecting the starting configuration to the target configuration. For the new planner, objects are initially classified as \quotes{unknown} and can later be categorized as either \quotes{movable} or \quotes{unmovable}. The object type information is then used when constructing the configuration space for the newly proposed planning algorithm, configuration space consists of the conventional free- and unmovable- (or obstacle) space, and the newly proposed unknown- and movable space.\bs

To carry out the investigation, a mobile robot in a robot environment with movable and unmovable objects is created. The robot is given a task that involves relocating a subset of the objects in the robot environment through driving and nonprehensile pushing. The task can be broken down into individual subtasks that consist of an object and a target pose. Planning for a push or drive action occurs with the newly proposed planning algorithm that, if successful, completes a given task and populates the \acl{k-graph} with learned object information. Information that can be used to determine which objects to manipulate, and what strategy performs best to manipulate a specific object.\bs

By combining these three topics, the proposed framework shows better task execution thanks to experience gained. The proposed framework performs equivalent or better compared to the state-of-the-art frameworks that are specialized in only two out of three research topics~\cite{ellis_navigation_2022,sabbaghnovin_model_2021,scholz_navigation_2016,vega-brown_asymptotically_2020,wang_affordancebased_2020}.\bs

\begin{flushright}
{\makeatletter\itshape
    \@author\\
    Delft, \monthname{} \the\year{}
\makeatother}
\end{flushright}
