\chapter{Conclusions}%
\label{chap:conclusion}
This thesis aims to combine three topics in robotics; learning object dynamics, the \ac{NAMO} problem and nonprehensile pushing, into a single robot framework. The overwelming research in the individual topics stand is contrast to the sparse number of recent papers that combine the three aforementioned topics. This absence of research can be motivated by the emerged challenges when combining the three topics such as the uncertainty in plannign and the complexity class of the combined problem.\bs


This main research question, that is: \textit{How do learned objects' system models improve global task planning for a robot with nonprehensile push manipulation abilities over time?} is answered in \Cref{chap:results} that presents the results. In this chapter the proposed robot framework is tested by providing a robot with a multiple tasks in various environments. These tests point out that task execution improves when the classification of an object is known, and when experience is gathered and leveraged in the robot environment. 

A classification of an object improves the success rate of generated action sequences, mainly because, probabilistic action sequences that may involve pushing unmovable objects cannot occur when object classification is present. During task execution the proposed framework gathers experience by storing action feedback after successful or unsuccessful actions. As a result the proposed framework converges toward an best strategy to manipulate objects. The best strategy consists of the best combination of controller and system model that yield desirable metrics (success rate and \acl{PE}) out of the available combinations of controllers and system models. \todo{Do you want to include path planning improves because of feasibility is checked by system models that act as a local planner?}

Research subquestion~\ref{researchsubquestion:does_it_work}, that is: \textit{How to combine learning and planning for push and drive applications?} is answered in \Cref{chap:required_background,chap:proposed_planning,chap:hgraph_and_kgraph}. Here it is shown how the proposed framework that consists of the \acl{halgorithm}, the \acl{hgraph} and the \acl{kgraph} work together to combine learning and planning with the technique of backward search. The curse of dimensionality and an \ac{NP-hard} problem are prevented by searching only in a single mode of dynamics instead of dicretly in the composite space.
\todo{rewreite below}

Where a search is performed for a drive or push action. The \ac{hgraph} is a graph-based structure that presents how the \ac{halgorithm} is trying to complete a subtask. The structure of the \ac{hgraph} is used to enforce the backward search technique, in which the \ac{halgorithm} searches from the target configuration toward the start configuration. Newly gained environmental knowledge is stored in the \ac{kgraph}, which firstly holds information on objects and if they can be manipulated, and secondly holds information on how they can be best manipulated. The \ac{kgraph} can be filled with newly learned environmental knowledge and can be queried for action suggestions.\bs

is answered in \Cref{chap:required_background,chap:hgraph_and_kgraph}. Here it has been shown that the proposed framework that consists of the \acl{halgorithm}, the \acl{hgraph} and the \acl{kgraph} work together can combine learning and planning with the technique of backward search. The curse of dimensionality and an \ac{NP-hard} problem are prevented by searching only in a single mode of dynamics instead of dicretly in the composite space.



Research subquestion~\ref{researchsubquestion:does_it_remember}, that is: \textit{
To what extend is the combination of the three topics; learning object dynamics, the \ac{NAMO} problem and nonprehensile pushing influenced by environmental experience?}

\todo{answer}


Research subquestion~\ref{researchsubquestion:does_it_compare}, that is: \textit{How does the proposed framework compare against the state-of-the-art?}

\todo{answer}
The proposed framework was able to perform as well, better or worse compared to the state-of-the-art, depending on which metric is taken into account. The proposed framework was tested against existing methods which only combine a subset of the 3 main topics that the proposed framework combines. 
The results discussed in \Cref{chap:results} are compared to the state-of-the-art papers in \Cref{sec:compare_with_related_papers}. Five state-of-the-art papers are used to compare against the proposed framework. In the comparison success rate, planning time, execution time and several calls to the planner are used to compare. From the results, it can be concluded that the proposed framework learns relatively fast (prediction error comparison with~\cite{wang_affordancebased_2020}). The proposed framework can complete all of its assigned subtasks, giving it a comparable success rate to the state-of-the-art (success rate comparison with~\cite{ellis_navigation_2022}). This thesis does not have small prediction errors or low final position errors, it can be seen that the prediction errors are worse compared to the state-of-the-art (final position errors comparison with~\cite{sabbaghnovin_model_2021}). \bs

The proposed framework improves upon the state-of-the-art exactly in its field of focus, improving task execution over time. It has been shown to learn faster compared to the state-of-the-art. Especially when the same or similar task is given to solve multiple times. Execution and planning times are equal or improved compared to the state-of-the-art. Lastly, prediction errors and final position errors are mainly worse compared to the state-of-the-art, which can be improved by better-designed controllers and more accurate system models. 



\section{Future work}%
\label{sec:future_work}
Possible extensions to the proposed framework and improvements on the proposed framework are grouped in this section.\bs

\subsection{Removing Assumptions}
Every assumption taken in \Cref{chap:introduction} serves to simplify the problem and to narrow the scope of this thesis. They can however all be removed. Starting with assumption \hyperref[assumption:closed_world]{\textbf{closed-world assumption}}, without this assumption the proposed framework must be much more robust. The proposed framework can with the help of this assumption conclude many conclusions deterministically. Examples are the feedback on edges and the classification of an object as unmovable or movable. In real-world applications unmovable objects can become movable, to make the transition toward removing the closed-world assumption the proposed framework must be converted to a probabilistic variant.\bs

Moving from simulation toward the real world must remove the \hyperref[assumption:perfect_object_sensor]{\textbf{perfect object sensor assumption}}. Sensors and sensor fusion is required to estimate the configuration of the robot itself and objects in the environment. A start is to test the proposed framework with noise added to the perfect sensor.\bs

The \hyperref[assumption:order_does_not_matter]{\textbf{tasks are commutative assumption}} assumption makes it possible to randomly select a subtask without influencing the feasibility of the task. Removing this assumption can require an additional rearrangement algorithm~\cite{krontiris_dealing_2015} to be run to determine a feasible order of handling subtasks.\bs

The \hyperref[assumption:no_tipping]{\textbf{objects do not tip over assumption}} prevents objects from tipping over, but at the same time limits the number of objects. There are many possibilities to handle tipped objects. First, adding a tipping detector can detect when an object has tipped over. Then the proposed framework can threaten the objects as a new object and reclassify it. Another method would be a dedicated subroutine to place it in an upright position.\bs

%
% \todo[inline]{The hgraph could think of new hypotheses during execution time if it would operate async, which would greatly improve it.}
% \todo[inline]{\cite{krontiris_dealing_2015} This thing could be calculated before everything, check the backward search you've commented out entirely}
\todo{system identificaiton instead of hard-coded models}
