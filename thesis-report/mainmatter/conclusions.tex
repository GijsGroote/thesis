\chapter{Conclusions}%
\label{chap:conclusion}
This thesis aims to create a robot framework that can; learn object dynamics, solve \ac{NAMO} problems and perform nonprehensile pushing. The overwhelming research in these individual topics stand is contrast to the sparse number of recent papers that combine these three topics. This absence of research can be motivated by the emerged challenges when combining the three topics, such as the uncertainty in planning and the complexity class of the overall problem.\bs

% main research question improved in two aspects
The main research question, that is: \textit{How do learned objects' system models improve global task planning for a robot with nonprehensile push manipulation abilities over time?} is answered in \Cref{chap:results} that presents the results. The results show that global task planning improves in two aspects as a result of learned system models. The first improvement is due to the classification of objects, which improves the success rate of generated action sequences. Mainly because probabilistic action sequences that may involve pushing unmovable objects cannot occur when object classification is present. The second improvement is due to selecting the best strategy to manipulate objects based on experience with such objects. Selecting the best strategy or edge parameterization can be summarized as follows. Initially, random edge parameterizations are selected from the available set of edge parameterizations. After execution of edges with such edge parameterization, the proposed framework gathers experience by storing action feedback. After all possible edge parameterizations are tested in the learning phase, the converging phase starts in which the proposed framework converges toward the best edge parameterization to manipulate a specific object.\bs

% randomized environment and comparison to state-of-the-art
Drive and pushing tasks have been performed with randomized settings to obtain unbiased results. In such a randomized environment, once tasks are solved whilst selecting action suggestions from the \ac{k-graph} if present, and once by only randomly selecting edge parameterizations from the set of available edge parameterizations. The results indicate that significant improvements can be made by leveraging environmental experience by selecting action suggestions. These improvements were measured with the \acl{PE} and the search-, execute- and total time to complete a task and investigating their development over time. Two essential factors determine convergence toward an optimal controller and system model combination selection. First, the success factor, because it determines which edge parameterization is the best candidate based on the metrics that are gathered during execution time of edges with such an edge parameterization. Second, the set of available controllers and system models, because the proposed framework can only converge to the best available controller in the set of available controllers.\bs

Research subquestion~\ref{researchsubquestion:does_it_work}, that is: \textit{How to combine learning and planning for push and drive applications?} An answer is given in \Cref{chap:required_background,chap:proposed_planning,chap:h-graph_and_k-graph}. Here it is shown how the proposed framework, which consists of the \acf{h-algorithm}, the \acf{h-graph} and the \acf{k-graph} work together to combine learning and planning with the backward search technique. The curse of dimensionality and an \ac{NP-hard} problem are bypassed by searching only in a single mode of dynamics instead of searching in composite space. \Cref{chap:required_background,chap:proposed_planning} present methods and functions utilized by the proposed robot framework. \Cref{chap:h-graph_and_k-graph} defines the proposed robot framework, in this chapter the \ac{h-graph} is defined as a graph-based structure that represents a search in composite space. The \ac{h-algorithm} generates action sequences by starting a search from the desired outcome and searching backwards to the current configuration with the backward search technique. Newly gained environmental knowledge is stored in the \ac{k-graph}, which firstly classifies objects as movable or unmovable, and, secondly, holds information on how they can be best manipulated. The \ac{k-graph} can be filled with newly learned environmental knowledge and can be queried for action suggestions. It is shown how learned system knowledge can be utilized by alternating between searching for an action sequence and executing an action sequence.\bs

Research subquestion~\ref{researchsubquestion:does_it_compare}, that is: \textit{How does the proposed framework compare against the state-of-the-art?} Only one state-of-the-art method was compared, whilst five had been selected for comparison. An improvement was recorded for search- and execution times when comparing the proposed method to \citeauthor{wang_affordancebased_2020} method. However, more comparisons with the state-of-the-art should be made to conclude improvements of the proposed method for other test metrics such as success rate or number of replanning. Thus more testing (and an system identification module) is required to conclude that the proposed method that can combine firstly: learning object dynamics, second: nonprehensile pushing, in thirdly: \ac{NAMO} problems is an improvement compared to the state-of-the-art.\bs

The proposed robot framework tackles \ac{NAMO} problems by incorporating a proposed path planner that detects blocked paths. The proposed framework includes nonprehensile pushing by generating action sequences that consist of drive and push actions. Thereby, \ac{NAMO} problems and nonprehensile pushing are combined in the proposed robot framework. Since the proposed framework classifies objects and selects the best available strategy, it improves its generated action sequences over time but does not learn object dynamics. During testing, analytic models were used that cannot model a wide variety of systems accurately. \Cref{table:sota_vs_results_proposed} shows, therefore a \xmark/\cmark, concluding that the proposed framework can partly combine the three topics. The analytic models could be replaced by adding a system identification module that is moved to the future work section, if implemented, the proposed framework can claim that the three topics can fully be combined.\bs
