\section{Control Methods}%
\label{sec:control_methods}
This section elaborates on why control is required and which control methods are best suitable for various control applications. Predictive control methods have been selected from many control methods because they heavily rely on a dynamic model of the system they control. During this thesis, the effect of the robot interacting with objects is captured by dynamic system models. In addition to predicting output with system models, predictive control methods use the system models to determine action input. A requirement for a controller is that it should yield a stable closed-loop control because that guarantees converging toward a set point. In the proposed framework, controllers are later selected for yielding desired metrics grouped in \Cref{sec:monitoring_metrics}. The two control methods that are used during testing are discussed below.\bs

\paragraph{\acl{MPC}}
The basic concept of \ac{MPC} is to use a dynamic model to forecast system behaviour and optimize the forecast to produce the best decision for the control move at the current time. Models are central to every form of \ac{MPC}~\cite{rawlings_model_2020}. The best feasible input is found every time step by optimizing the objective function and respecting constraints. Tuning is accomplished by modifying the objective function's weight matrices or the constraint set. Minimizing an objective function to find the best feasible input generally yields robust control. A more elaborate description of \ac{MPC} control can be found in the appendix.\bs

\paragraph{\acl{MPPI} Control}
The core idea is from the current state of the system using a system model and randomly sampled inputs to simulate several \quotes{rollouts} for a specific time horizon,~\cite{neuromorphictutorial_ltc21_2021}. These rollouts indicate the system's future states if the randomly sampled inputs are applied. The future states can be evaluated by a cost function penalizing undesired states and rewarding desired future states. A weighted sum over all rollouts determines the input which will be applied to the system. The main advantage \ac{MPPI} has over \ac{MPC} control is that it is better suited for nonlinear system models. Whilst linear models can accurately estimate drive applications, push applications are harder to estimate with a linear model. Thus \ac{MPPI} is selected mainly for push applications. A more elaborate description of \ac{MPPI} control can be found in \Cref{sec:appendix_mppi}.\bs

The properties of \ac{MPC} suggest that it is best suitable for drive actions because of easy tuning and robustness. \Ac{MPPI} control is compatible with nonlinear system models, making it more suitable for push actions. It is worth mentioning that the goal of this thesis is not to find the best optimal controller. The goal is to gradually, over time, choose control methods in combination with system models that result in better performance, and the performance is measured with various metrics, discussed in \Cref{chap:hgraph_and_kgraph}.\bs


% Not all control methods are compatible with every system identification method, the following table conveniently displays which control methods are compatible with the system identification methods that are implemented during the thesis.\bs
%
% \begin{figure}[H]
% \begin{minipage}{0.5\linewidth}
% \begin{table}[H]
% \centering
% \begin{tabular}[t]{l c c}
%   control sys.~iden. & \acs{PEM} & \acs{IPEM} \\
%   \toprule
%   \ac{MPC} & \cmark& \cmark\\
%   \ac{MPPI} & \cmark& \cmark\\
% \end{tabular}
% \caption{Drive Control}%
% \label{table:compatible_modules_drive}
% \end{table}
% \end{minipage}
% \begin{minipage}{0.5\linewidth}
% \begin{table}[H]
% \centering
% \begin{tabular}[t]{l c c }
%   control sys.~iden. & \acs{PEM} & \acs{LSTM}\\
%   \toprule
%   \ac{MPC} & \cmark& \xmark\\
%   \ac{MPPI} & \cmark& \cmark\\
% \end{tabular}
% \caption{Push Control}%
% \label{table:compatible_modules_push}
% \end{table}
% \end{minipage}
% \caption{Compatibility between control and system identification methods for drive and push control.}%
% \label{table:compatible_modules}
% \end{figure}
