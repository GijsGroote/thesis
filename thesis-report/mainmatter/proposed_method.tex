\chapter{Planning in Four Subspaces}%
\label{chap:proposed_planning}
\textit{Finding a path between the start- and target configuration whilst avoiding collisions with an existing \ac{RRT*} path planner~\cite{chen_fast_2018} was presented at the end of the previous chapter. This chapter extends that existing path planner's configuration space from free- and obstacle space to also incorperate movable- and unknown space. Movable space originates from objects in the environment that are known to be movable, unknown space originates from objects for which it is unknown weither they are movable or unmovable. The planner avoids obstacle space, is incentivezed to plan only in free space but can pass through unknown or movable space. A direct path lies in free space only, a blocked path passes through unknown- or movable space. To free a blocked path and make it a direct path, the object that causes the unknown- or movable space should be moved out of the way.\bs}

The modified algorithm has four tuning parameters that can be tweaked; The \textit{step size} and \textit{search size} were discussed in \Cref{subsec:motion_planning}. The third and fourth tuning parameters are the \textit{UnknownSpaceCost} and \textit{MovableSpaceCost}, which are fixed costs for crossing through unknown or movable space. Crossing through is defined as one or more nodes in the path lying in that subspace. If a path does not contain a node in unknown space, \textit{UnknownSpaceCost} will be 0, equivalent to movable space and \textit{MovableSpaceCost}. These cost are added to the $\mathit{Cost_{path}}$, which is redefined as:\bs

\[\mathit{Cost_{path}} = \mathit{MovableSpaceCost} + \mathit{UnknownSpaceCost} + \sum_{i=1}^{n-1} \mathit{Distance}(\gls{c}_i, \gls{c}_{i+1})\]

Where $\mathit{n} \geq 2$ configuration points make up the path starting at $\gls{c}_\mathit{start}$ and ending at $\gls{c}_\mathit{target}$.\bs

The \ac{RRT*} algorithm searches for a path with the lowest cost, by adding a penalty for crossing through unknown or movable space the path planner is incentivised to find the shortest path around objects but prefers moving an object over making a large detour. Tuning the additional fixed cost for a path crossing through movable or unknown space balances the robot's decision between the length of a detour the robot is willing to drive, compared to pushing an object to free the path. Removing an unknown object bears more uncertainty than a movable object, motivating a higher cost to remove an unknown object compared to an known object. The pseudocode from \Cref{pseudocode:proposed_rrt_star_all} is presented again, with changes due to the extension indicated with a red colour.\bs

\newpage
\begin{algorithm}[H]
  \caption{Pseudocode for extended \ac{RRT*} path planning algorithm. Lines that contain changes compared to \Cref{pseudocode:proposed_rrt_star_all} are indicated with the red colour.}%
  \label{pseudocode:modified_proposed_rrt_star}
  \begin{algorithmic}[1]
    \State $\gls{nodesMP} \leftarrow x_{init}$
    \While{\textit{NotReachStop}}
        \State $\gls{nodeMP}_\mathit{rand} \leftarrow \mathit{Sample_{random}}$ \algorithmiccomment{Create, project and validate a new random sample}
      \State $\gls{nodeMP}_\mathit{nearest} \leftarrow \mathit{Nearest(\gls{nodeMP}_{rand}, \gls{nodesMP})}$
      \State $\gls{nodeMP}_\mathit{temp} \leftarrow \mathit{Project(\gls{nodeMP}_{rand}, \gls{nodeMP}_{nearest})}$
      \If{$\mathit{CollisionCheck(\gls{nodeMP}_{temp})}$}
      \State $\gls{nodeMP}_\mathit{new} = \gls{nodeMP}_\mathit{temp}$
      \State $\mathit{Cost_{toInitMin}} \leftarrow +\infty$ 
      \Else
      \State Continue
      \EndIf
      \State $X_\mathit{near} \leftarrow \mathit{NearestSet(\gls{nodeMP}_{new}, \gls{nodesMP})}$ \algorithmiccomment{Find and connect new node to parent node}
      \For{$\gls{nodeMP}_\mathit{near} \in X_\mathit{near}$}
    \State $\mathit{Cost_{temp}} \leftarrow \mathit{CostToInit}(\gls{nodeMP}_\mathit{near}) + \mathit{Distance}(\gls{nodeMP}_\mathit{near}, \gls{nodeMP}_\mathit{new}) + \textcolor{red}{\mathit{ObjectCost}(\gls{nodeMP}_\mathit{near}, \gls{nodeMP}_\mathit{new})}$
      \If{$\mathit{Cost_{temp}}  < \mathit{Cost_{toInitMin}}$}
      \State $\mathit{Cost_{toInitMin}} \leftarrow \mathit{Cost_{new}}$
      \State $\gls{nodeMP}_\mathit{minCost} \leftarrow \gls{nodeMP}_\mathit{near}$
      \EndIf
      \EndFor
      \If{$\mathit{Cost_{toInitMin}} == \infty$}
          \State Continue
      \Else
      \State $\gls{nodesMP}.add(\gls{nodeMP}_\mathit{new})$
      \State $E.\mathit{add}(\gls{nodeMP}_\mathit{minCost}, \gls{nodeMP}_\mathit{new})$
      \EndIf
      \State $\mathit{Cost_{pathMin}} \leftarrow +\infty$ 
      \For{$\gls{nodeMP}_\mathit{near} \in X_{near}$}\algorithmiccomment{\parbox[t]{.6\linewidth}{Check if the newly added node can lower cost for nearby nodes and if a both connectivity trees can be connected}}
      \If{$\mathit{InSameTree(\gls{nodeMP}_{near}, \gls{nodeMP}_{new})}$}
    \If{$\mathit{CostToInit(\gls{nodeMP}_{new})} + \mathit{Distance(\gls{nodeMP}_{new}, \gls{nodeMP}_{near})} +\textcolor{red}{\mathit{ObjectCost}(\gls{nodeMP}_\mathit{new}, \gls{nodeMP}_\mathit{near})} < \mathit{CostToInit(\gls{nodeMP}_{near})}$}
      \State $\mathit{E.rewire(\gls{nodeMP}_{near}, \gls{nodeMP}_{new})}$
      \EndIf
      \Else \algorithmiccomment{Add lowest cost path to the list of paths}

      \State $\mathit{Cost_{temp} \leftarrow CostToInit(\gls{nodeMP}_{new}) + Distance(\gls{nodeMP}_{new}, \gls{nodeMP}_{near})} + \mathit{CostToInit(\gls{nodeMP}_{near})}$
      % \State $\mathit{Cost_{temp} \leftarrow CostToInit(\gls{nodeMP}_{new}) + Distance(\gls{nodeMP}_{new}, \gls{nodeMP}_{near})} + \mathit{CostToInit(\gls{nodeMP}_{near}) + \textcolor{red}{ObjectCost(\gls{nodeMP}_{new}, \gls{nodeMP}_{near})}}$
      \If{$\mathit{Cost_{temp}  < Cost_{pathMin}}$}
      \State $\mathit{Cost_{pathMin}} \leftarrow \mathit{Cost_{temp}}$
      \State $\mathit{\gls{nodeMP}_{pathMin} \leftarrow \gls{nodeMP}_{near}}$
      \EndIf
      \EndIf
      \If{$Cost_{pathMin} == \infty$}
      \State Continue
      \Else
      \State $\mathit{P.addPath(\gls{nodeMP}_{new}, \gls{nodeMP}_{pathMin}, Cost_{pathMin})}$
      \EndIf
      \EndFor
    \EndWhile
  \end{algorithmic}
\end{algorithm}

The proposed algorithm prevents planning a path through blocking objects except when no other option is available or a large detour can be prevented. No performance tests have been conducted on the modified path planner, apart from visual inspection. \Cref{fig:mp_push_or_drive} clearly shows the effect of varying \textit{UnknownSpaceCosts}.

\begin{figure}[H]
    \centering
    \begin{subfigure}{\textwidth}
    \centering
  \includegraphics[width=0.8\textwidth]{figures/proposed_method/push_or_drive}
    \caption{Robot environment with the point robot, two yellow unmovable walls and an unknown brown box.\\The robot tasked to drive toward the opposite side of the brown box.}\label{subfig:push_or_drive_env}
    \end{subfigure}

    \begin{subfigure}{1.11\textwidth}
    \centering
    \includegraphics[width=\textwidth]{figures/required_background/mp/mp_high_fixed_cost}
    \caption{Visualization of the planned path around the brown box and yellow obstacles, with $\mathit{UnknownSpaceCost} = 1$.}
    \end{subfigure}

    \begin{subfigure}{1.11\textwidth}
    \centering
    \includegraphics[width=\textwidth]{figures/required_background/mp/mp_low_fixed_cost}
    \caption{Visualization of the planned path going through the brown box, $\mathit{UnknownSpaceCost} = 0.5$.}
    \end{subfigure}
    \caption{Driving task and two planned paths.}%
    \label{fig:mp_push_or_drive}
\end{figure}


\section{Kinodynamic Planning}
A generated path for a non-holonomic robot, such as the boxer robot in \Cref{subfig:example_boxer_robot} should respect the non-holonomic constraints. To take dynamic constraints (e.g. non-holonomic constraints or constrains on the derivatives of the planned path) into account, a dynamic model should be incorperated. Respecting dynamic constraints during path planning is named \textit{kinodynamic planning} 

\todo{cite some kinodynamic sources}
\todo{eleboarte on kinodyn and the sample based}
\todo{eleborate on how to incoreperate a dynamic model (and that sampling is then a bit different)}










Now that the modified path planner is discussed, the proposed robotic framework is discussed. The proposed framework relies on the required background from previous chapter, and relies on the modified path planner from this chapter.\bs



% old manipulation section shit

% \subsection{Manipulation Planning}%
% \label{subsec:manipulation_planning}
% With a push, two objects are primarily involved, the pushed object and the robot. Generally, and in this thesis, the pushed object's configuration is more important than the robot's configuration. The robot is only a means to push the object toward the target configuration. At which final configuration the robot itself ends up is of lesser importance. As long as during the push, the robot does not collide with objects other than the pushed object, and constraints on the robot must be respected.\bs

% To plan a path that respects the constraints, \todo{Corrado: What does this mean? ->} the robot's configuration is generated for every newly added sample in the manipulation planning algorithm.

% \todo{Gijs For reachbailbpcheck, These lines just point out the name of the function again , there is no added information }
% The \textit{ReachabilityCheck()} (see \Cref{table:functions_for_proposed_rrt_star} and line 33 in \Cref{pseudocode:proposed_rrt_star}) generates the robot configuration to validate if a new sample is reachable from an existing sample. This additional configuration is stored to create only feasible paths that respect the applied constraints. When the stopping criteria are reached and the shortest path is found, the generated robot configurations are discarded. \Cref{fig:manipulation_plannig_local_planner} displays a visual example of the procedure.\bs

% \begin{figure}[H]
%     \centering
%     \includegraphics[width=0.6\textwidth]{figures/required_background/manipulation_local_planner}
%     \caption{Generating a new robot configuration whilst adding a sample to the connectivity tree during manipulation planning.}%
%     \label{fig:manipulation_plannig_local_planner}
% \end{figure}

 
\chapter{Proposed Robot Framework}%
\label{chap:h-graph_and_k-graph}
\textit{This chapter is dedicated to introducing and defining the proposed framework. The proposed framework consists of the \acl{h-algorithm}, the \acl{h-graph} and the \acl{k-graph}. The \textbf{\acf{h-algorithm}} acts on the \textbf{\acf{h-graph}} and is responsible for searching and executing action sequences to complete a specified task. \Cref{sec:h-graph} is dedicated to introducing and defining the \ac{h-graph}. Then the \ac{h-algorithm} is discussed and defined in \Cref{sec:h-algorithm}. The chapter finalizes with the \ac{k-graph} in \Cref{sec:k-graph}.\bs}

\section{Overview of the Proposed Framework}
\Cref{tikz:flowchart_proposed_method} presents a schematic overview of the interconnection of the \ac{k-graph}, \ac{h-algorithm} and the robot environment.\bs

\vspace{-0.2cm}
\begin{figure}[H] \centering
\begin{tikzpicture}[node distance = 2cm, auto]
    % Place nodes
    \node[draw=gray, rounded corners, inner sep=3ex, line width=7pt, fill=gray, fill opacity=0.4, minimum height=11.0cm, minimum width=5.8cm, yshift=3.25cm] (focusbox) {};
    \node[yshift=5.8cm, xshift=-1.5cm, align=left] at (focusbox) {\textbf{Thesis focus}};

   \node [outer sep=0cm] (environment) at (0,0)  {\includegraphics[width=4.6cm]{figures/proposed_method/example_environment.png}};

   \node [below, xshift=0.4cm, yshift=-.1cm, text width=5cm, align=left, outer sep=0cm] at (environment.north) {\textbf{Robot Environment}};

    \draw [myEvenLighterColor,
    rounded corners=0.3cm,
    line width=0.3cm]
    (environment.north west) --
    (environment.north east) --
    (environment.south east) --
    (environment.south west) -- cycle  ;

    \node [block,
    above of=environment,
    minimum height=2cm,
    minimum width=5cm,
    node distance=4.1cm,
    outer sep=0cm] (h-graph) {Hypothesis Algorithm};

    \node [block,
    above of=h-graph,
    node distance=3.3cm,
    minimum width=5cm,
    minimum height=2.0cm] (k-graph) {Knowledge Graph};

    \node [rectangle, draw,
    fill=myEvenLighterColor,
    text width=5em, text centered, rounded corners,
    right of=k-graph,
    minimum width=4cm,
    minimum height=2cm,
    node distance=7.2cm] (ontology) {Ontology};

    \node [rectangle, draw,
    fill=myEvenLighterColor,
    text width=5em, text centered, rounded corners,
    right of=h-graph,
    minimum width=4cm,
    minimum height=2cm,
    node distance=7.2cm] (planner) {High-level planner};

    % Draw edges
    \draw[-stealth] ([yshift=0.155cm, xshift=0.4 cm]environment.north) -- node [xshift=-.05cm, right] {\shortstack[]{sensor\\measurements}}([xshift=0.4 cm]h-graph.south) ;
    \draw[-stealth] ([xshift=-0.4 cm]h-graph.south) -- node [left] {robot input}([yshift=0.155cm, xshift=-0.4 cm]environment.north) ;
    \draw[-stealth] (planner.west) -- node [pos=0.37, above] {task}(h-graph.east);
    \draw[-stealth] ([xshift=-0.4cm]k-graph.south) -- node [left] {\shortstack[]{action\\suggestions}}([xshift= -0.4cm]h-graph.north) ;
    \draw[stealth-] ([xshift=0.4cm]k-graph.south) -- node [right] {\shortstack[]{action\\feedback}}([xshift= 0.4cm]h-graph.north) ;
    \draw[-stealth] (k-graph.east) -- node [xshift=-0.1cm, above, pos=0.63] {\shortstack[]{environment\\knowledge}}(ontology.west);
    \draw[stealth-] ([xshift=0.4cm]ontology.south) -- node [right] {\shortstack[]{query}}([xshift=0.4cm]planner.north);
    \draw[-stealth] ([xshift=-0.4cm]ontology.south) -- node [left] {\shortstack[]{output}}([xshift=-0.4cm]planner.north);
    \draw[stealth-] (planner.south) |- ++ (2,-1) node[near end, above] {\shortstack[]{High-level\\task}};
    \end{tikzpicture}
\caption{Flowchart representation of the proposed robot framework.}%
\label{tikz:flowchart_proposed_method}
\end{figure}

The above figure shows that the thesis focus could be augmented with an ontology and high-level planner. Such an augmentation would create a framework capable of completing high-level tasks such as cleaning or exploring.\bs

\section{Hypothesis Graph}%
\label{sec:hgraph}
The \ac{hgraph} is responsible for generating action sequences, called hypothesis (consisting of a list of succesive edges from start to target node in the \ac{hgraph}) that complete a single subtasks. When all subtasks in a task are completed, the \ac{hgraph} halts and concludes the task succesfully completed. A search in the joint configuration space is avoided because an edge only operates in a single mode of dynamics, such as driving or pushing. When a object cannot directly be steered toward it's target location new nodes are generated which need to be completed before the original object can be steered toward it's target location. An example of when an object cannot directly be steered toward it's target state is because the path is blocked by another object. An hypothesis, consisting of a list of edges that represent actions in the robot environment might succeed or fail. \Cref{tikz:flowchart_hgraph} displays a flowchart explaining how new nodes and edges are generated in the \ac{hgraph}. Succesfully completed edges eventually result in completed subtasks, failed edges trigger replanning that will restart the search to a hypothesis.\bs

\Cref{subsec:hgraph_definition} defines the \ac{hgraph} and components, then~\cref{subsec:two_loops} eleborates how the \ac{hgraph} searches for a solution in the joint configuration space. Followed by subsections elaborating the methods used by the \ac{hgraph} such as system identification and control methods in \cref{subsec:sys_iden_and_control_methods}, path estimation in \cref{subsec:estimating_path_existence}, motion planning in \cref{subsec:motion_planning}, manipulation planning in \cref{subsec:manipulation_planning} and fault detection in \cref{subsec:fault_detection}.\bs

% For every subtask in a task, a start and a target node is created, the hypothesis graph tries to connect the starting node to the corresponding target node by adding node and edges. The choice and content of these nodes and edges is based on local planning and randomisation, elaborated in~\cref{subsec:estimating_path_existence,subsec:motion_planning,subsec:manipulation_planning}. A successive path from a starting node to the corresponding target node is called a hypothesis. During the search for a hypothesis, the hypothesis graph resides in the \textbf{search loop}, when traversing over edges toward the target node the hypothesis graph resides in the \textbf{execution loop}, both loops are elaborated in \cref{subsec:2_loops}. When the hypothesis graph traverses over an edge, fault detectors monitors the progress, elaborated in \cref{subsec:fault_detection}. Finally an example of an hypothesis graph is given in \cref{subsec:hgraph_example}.\bs
%
% After execution, the traversed edge enters a review period, during which performance is checked in various metrics. The review is stored in a database, called the \textbf{knowledge graph} which serves to collect edges and rank them based on performance. The knowledge graph is defined in \cref{subsec:kgraph_definition}, metrics to rank edges can be found in \cref{subsec:edge_metrics} and an example is displayed in \cref{subsec:kgraph_example}.
% \todo[inline]{come back to joint configuration space, and tell the hgraph cuts the different modes of dynamics with an edge for every mode of dynamics}


\subsection{Definition}
\label{subsec: hgraph_definition}

\todo[inline]{definition of state, node, }



\textbf{Lifetime and types of edges}
\newline
An edge denotes that input commands are sent to the robot and as a result the robot can drive around changing it's state. Obstacles in a certain state represented by a node in the hgraph can be connected to other nodes via edges.
The edge contains all necessary components to sent input to the robot resulting in an obstacle reaching the target state residing in the node where the edge points toward. \\

There exist two types of edges, identification edges and action edges. An identification edge is responsible for sending an input sequence to the system and recording the system output which the simulation returns. An input/output sequence and assumptions on the system are the basis for system identification techniques discussed in \cref{subsec: sys_iden_and_control_methods}. The goal is to create a dynamical model which if augmented with an corresponding controller is closed-loop stable. \\

An action edge is responsible for putting an obstacle at a target state, the goal is to reach the target state, without triggering the fault detector and scoring well at internal metrics elaborated in \cref{subsec: edge_metrics}.

\todo[inline]{the lifetime of an identification edge}

The different status an edge can have is listed below. An edge status can only change according to the directed arrows in \cref{tikz: lifetime_action_edge}

\definecolor{lavenderIndigo}{RGB}{160, 110, 224}
\definecolor{batteryChargedBlue}{RGB}{22, 164, 216}
\definecolor{skyBlue}{RGB}{96, 219, 232}
\definecolor{kiwi}{RGB}{139, 211, 70}
\definecolor{minionYellow}{RGB}{239, 223, 72}
\definecolor{deepSaffron}{RGB}{249, 165, 44}
\definecolor{sinopia}{RGB}{214, 78, 18}

\begin{figure}[H]
\centering
\begin{tikzpicture}[node distance = 2cm, auto]
    \node [block, fill=lavenderIndigo] (init) {Initialised};
    \node [block, fill=batteryChargedBlue, below of=init] (path_exist) {Path Exists};
    \node [block, fill=skyBlue, below of=path_exist] (system_model) {System Model};
    \node [block, fill=kiwi, below of=system_model] (path_planned) {Path Planned};
    \node [block, fill=minionYellow, below of=path_planned] (executing) {Executing};
    \node [block, fill=deepSaffron, below of=executing] (completed) {Completed};
    \node [block, fill=sinopia] (failed) at ([xshift=4cm]$(system_model)!0.5!(path_planned)$) {Failed};
    
    % arrows
    \draw [-stealth] ([xshift=-2cm]init.west) to node[near start,above]{select controller} (init.west);
    \draw [-stealth] (init.south) to node[left]{graph-based path estimation} (path_exist.north);
    \draw [-stealth] (path_exist.west) [out=215,in=145] to node[left]{system identification} ([yshift=0.3cm] system_model.west);
    \draw [-stealth] ([yshift=-0.3cm] system_model.west) [out=215,in=145] to node[left]{motion planning} ([yshift=0.3cm] path_planned.west);
    \draw [-stealth] ([yshift=-0.3cm] path_planned.west) [out=215,in=145] to  node[left]{go to execution loop} ([yshift=0.3cm] executing.west);
    \draw [-stealth] (executing.south) to node[left]{evaluate with metrics} (completed.north);
    % edges to failed
    \draw [-stealth] (init.east) [out=0, in=90] to node[right]{path non-existence proven}  ([xshift=0.3cm]failed.north);
    \draw [-stealth] (path_exist.east) [out=0, in=90] to node[xshift=-0.4cm,yshift=0.55cm, above]{\shortstack[l]{system\\identification\\error}}  ([xshift=-0.3cm]failed.north);
    \draw [-stealth] (system_model.east) [out=0, in=180] to node[xshift=0.1cm, yshift=0.3cm, above]{\shortstack[l]{motion\\planning\\error}} ([yshift=0.3cm]failed.west);
    node[right]{motion planning error}  
    ([yshift=-0.3cm]failed.west);
    \draw [-stealth] (executing.east) [out=0, in=-90] to node[right]{fault detected}(failed.south);
\end{tikzpicture}
\caption{The status of an action edge and the most important indicator to change status}
\label{tikz: lifetime_action_edge}
\end{figure}

\begin{enumerate}
    \item[INITIALISED] The edge is created with a source and target node which are present in the hypothesis graph. A choice of controller is made.
    \item[PATH EXISTS] A graph-based search is performed to validate if the target state is reachable assuming that the system is holonomic.
    \item[SYSTEM MODEL] A dynamics system model is provided to the controller residing in the edge.
    \item[PATH PLANNED] Resulting from a sampble-based planner, a path from start to target state is provided. 
    \item[EXECUTING] The edge is currently sending input toward the robot. 
    \item[COMPLETED] The edge has driven the system toward it's target state and it's performance has been calculated.
    \item[FAILED] An error occurred, yielding the edge unusable. 
\end{enumerate}
\subsection{The Search and the Execution loop}%
\label{subsec:two_loops}
\todo[inline]{introduce search and execution loop}
\todo[inline]{Adding nodes always happens, in a backward search fashion}

\newpage
\newgeometry{left=1.1cm,bottom=0.1cm,top=1.9cm,headsep=0.1in,heightrounded}

\begin{figure}[H] 
\centering
\begin{tikzpicture}[node distance = 3cm]
    % Nodes
    \node [block, fill=yellow!50, line width=2pt, dashed] (first) {create start and target nodes};
    
    % legend
    \node[text width=2.8cm, yshift=1cm, right of=first, node distance=7cm, text centered, rounded corners, minimum height=1em, label={[name=lab, yshift=0.4cm, left]\textbf{Legend}}] (legend1) {\small Update KGraph};
    \node[rectangle, draw, left of=legend1, fill=green!50, rounded corners, minimum height=1em, minimum width=1cm, node distance=2cm] (legend1color) {};
    
    \node[text width=2.8cm, below of=legend1, text centered, minimum height=1em, node distance=0.7cm] (legend2) {\small Query KGraph};
    \node[rectangle, draw, left of=legend2, fill=red!40, rounded corners, minimum height=1em, minimum width=1cm, node distance=2cm] (legend2color) {};
   
    \node[text width=2.8cm, below of=legend2, text centered, minimum height=1em, node distance=0.7cm] (legend3) {\small Update C-Space};
\node[rectangle, draw, left of=legend3, fill=yellow!50, rounded corners, minimum height=1em, minimum width=1cm, node distance=2cm] (legend3color) {};
    
    \node[text width=2.8cm, below of=legend3, text centered, minimum height=1em, node distance=0.7cm] (legend4) {\small action in HGraph};
    \node[rectangle, draw, left of=legend4, rounded corners, minimum height=1em, minimum width=1cm, node distance=2cm, line width=2pt, dashed] (legend4color) {};
 
    \node[text width=2.8cm, below of=legend4, text centered, minimum height=1em, node distance=0.7cm] (legend5) {\small action in C-Space};
\node[rectangle, draw, left of=legend5, rounded corners, minimum height=1em, minimum width=1cm, node distance=2cm, line width=2pt] (legend5color) {};

    % nodes, Path exists 
    \node [decision, below of=first, node distance=2.6cm, line width=2pt] (path_existence) {Estimate Path Existence};
    \node [decision, left of=path_existence, node distance=4.5cm, line width=2pt, dashed] (subtasks) {Is there an unfinished Subtask};
    \node [block, above of=subtasks, node distance=2.7cm] (no_solution_found) {no solution found};
    
    % nodes, Knowledge available
    \node [decision, fill=red!40, below of=path_existence, node distance=3.2cm, inner sep=0.5mm] (know_avail) { Knowledge Available };
    \node [decision, fill=red!40, right of=know_avail, node distance=3.5cm, inner sep=0.5mm] (know_good) {Knowledge Usable};
    \node [decision, right of=know_good, node distance=3.5cm, inner sep=0.5mm] (movable) {Object Movable or Unknown};
    \node [block, left of=know_avail, node distance=3cm, line width=2pt, dashed] (impossible) {impossible task, abort subtask};
    
    % nodes, Generate new edge
    \node [decision, below of=know_avail, node distance=3.2cm, line width=2pt, inner sep=0.5mm, dashed] (goto_sys_iden) {Generate Random Edge};
    \node[block, right of=goto_sys_iden, node distance=3.5cm, line width=2pt, dashed] (no_trans_found) {No more edges available, abort subtask};
    
    
    % Motion/Manipulation planning 
    \node [decision, below of=goto_sys_iden, node distance=3.5cm] (single_multi) {Driving or Pushing?};

    \node [decision, line width=2pt, dashed, left of=single_multi, node distance=3.7cm] (model_avail_single) {Model available};
    \node [decision, line width=2pt, dashed, left of=single_multi, node distance=3.7cm] (model_avail_single) {Model available};
    \node [decision, line width=2pt, dashed, right of=single_multi, node distance=3.7cm] (model_avail_multi) {Model available};
    \node [block, line width=2pt, dashed, left of=model_avail_single, node distance=2.7cm] (sys_iden_single) {Add Sys. Iden. Node};
    \node [block, line width=2pt, dashed, right of=model_avail_multi, node distance=3cm] (sys_iden_multi) {Add Sys. Iden. Node};
    \node [block, line width=2pt, dashed, below of=single_multi, node distance=3cm] (move_object) {Add Node to Move Object};
    \node [block, line width=2pt, left of=move_object, node distance=3.7cm] (motion_planning) {Motion Planning};
    \node [block, line width=2pt, right of=move_object, minimum width=2.3cm, node distance=3.7cm] (manipulation_planning) {Manipulation Planning};
    \node [block, line width=2pt, dashed, minimum width=2.3cm, below of=move_object] (drive_to_object) {Add drive subtask to object};

    % nodes, Path to target
    \node [decision, below right of=drive_to_object, node distance=4.0cm, line width=2pt, dashed] (first_action) {First Action Planned};
    \node [decision, below left of=drive_to_object, node distance=3.5cm, line width=2pt, dashed] (global_path) {Path to Target}; 

    % \node [block, line width=2pt, dashed, minimum width=2.3cm] (drive_to_object) at ([xshift=0.1cm]$(move_object)!0.5!(global_path)$) {Add drive subtask to object};
    \node [decision, right of=first_action, diagonal fill={yellow!50}{green!50}, node distance=3cm] (execute) {Execute};
     
    % nodes, Target node reached 
    \node [decision, below of=global_path, node distance=3cm, line width=2pt, dashed] (target_node_reached) {Target Node Reached};
    \node [block, left of=target_node_reached, node distance=3cm] (end) {Task successfully executed};
    
    % Edges
    \path[line] ++(0,1.5) -- node[left]{task} (first);
    \path[line] (first) -- node[midway](to_path_exists){}(path_existence); 
    
    % edges, Path exists 
    \path[line] (path_existence) -- node[midway, above, left] {No path found} (impossible.north east);
    \path[line] (subtasks.north) --  node[left] {no} (no_solution_found);
    \path[line] (path_existence) -- node[xshift=0.08cm, yshift=0.35cm, right] {path found} (know_avail); 
    \path[line] (subtasks.east) -- node[above] {yes} (path_existence.west);
    
    % edges, Knowledge available
    \path[line] (know_avail) -- node[above] {yes} (know_good); 
    \path[line] (know_good) -- node[yshift=0.1cm, above] {no} (goto_sys_iden); 
    \path[line] (know_avail) -- node[left](toward_new_trans) {no} (goto_sys_iden); 
    \draw[->] (know_good.east) -- node[above] {yes} (movable.west);
    
    % \draw[-]  ([xshift=3.2mm]toward_new_trans.center) -| node[near start, above] {no} (know_good.south);
    \draw[-](impossible.west) -- +(-0.47,0); 
     
    \draw[->]  ([xshift=1.75cm, yshift=7.3cm]know_avail.center) --  node[at start, above] {action suggestions} ([xshift=1.75cm, yshift=1.75cm]know_avail.center) -- (know_avail.north east);
    \draw[->]  ([xshift=1.75cm, yshift=1.75cm]know_avail.center) -- (know_good.north west);
    \draw [draw=white,double distance=\pgflinewidth,ultra thick] (path_existence.east) -- +(2cm,0);
    
    % edges, Generate new edge
    \draw[-] (move_object.south) |- +(-8.2,-0.3);
    \draw [draw=white,double=black,double distance=\pgflinewidth,ultra thick] (motion_planning.south) -- +(0,-1cm);
    \draw[-stealth] (motion_planning.south)  -- ([yshift=-1cm]motion_planning.south) -| node[near start, above] {success} (global_path.north);
    \draw[-stealth] (manipulation_planning.south) |- node[near start, right] {success} (drive_to_object.east);
    \draw[-] (drive_to_object.west) -| (global_path.north);
    \draw[-] (motion_planning.west) -- node[above] {failure} +(-3.47,0);
    \draw[-] (manipulation_planning.east) -| node[near start, above] {failure} ([xshift=4.7cm,yshift=-0.6cm]no_trans_found.south) -- ([yshift=-0.6cm]no_trans_found.south);
    
    % edges, Single/Multi body
    \draw[-stealth] (single_multi.west) -- node[above] {driving} (model_avail_single);
    \draw[-stealth] (single_multi.east) -- node[above] {pushing} (model_avail_multi);
    \draw[-stealth] (model_avail_single.south) -- node[left] {yes} (motion_planning.north);
    \draw[-stealth] (model_avail_single.west) -- node[above] {no} (sys_iden_single);
    \draw[-stealth] (model_avail_multi.south) -- node[near start, left] {yes} (manipulation_planning.north);
    \draw[-stealth] (model_avail_multi.east) -- node[above] {no} (sys_iden_multi);
    \draw[-stealth] (motion_planning.east) -- node[above] {blockade} (move_object);
    \draw[-stealth] (manipulation_planning.west) -- node[above] {blockade} (move_object);
    \draw[-stealth] (goto_sys_iden) -- node[above] {failure} (no_trans_found);
    \draw[-] (sys_iden_single.north) --  ([yshift=1.07cm]sys_iden_single.north);
    \draw[-] (sys_iden_multi.north) |-  ([yshift=-0.6cm]no_trans_found.south);
    \draw[-] (no_trans_found.south) -- ++(0,-0.6cm) --([xshift=-8cm, yshift=-0.6cm]no_trans_found.south);
    \draw [draw=white,double=black,double distance=\pgflinewidth,ultra thick] (goto_sys_iden.south) -- node[at start, left] {success}(single_multi.north);
    \draw[-stealth] ([yshift=-0.5cm]goto_sys_iden) -- (single_multi.north);
    
    \draw[-] (movable.south) |- node[near start, left] {yes} ([xshift=-1.5cm, yshift=-1.4cm]movable.south) |- ([yshift=0.2cm]single_multi.north);
    \draw [draw=white,double distance=\pgflinewidth,ultra thick]  ([xshift=-1cm]movable.north) -- ([xshift=-8cm]movable.north);

    \draw[-] (movable.north) -- node[above]{no}([xshift=-8.23cm]movable.north);
    % HERE
    \draw [draw=white,double=black,double distance=\pgflinewidth,ultra thick] ([xshift=5.5cm,yshift=0.3cm]single_multi.north) -- ([xshift=5.5cm, yshift=2cm]single_multi.north);
    % \draw[-] (know_good.east) -| node[above]{yes} ([xshift=5.5cm, yshift=0.2cm]single_multi.north) -- ([yshift=0.2cm]single_multi.north);
    
    % edges, Path to target
    \path[line] (global_path) -- node[above] {yes} (first_action);
    \path[line] (first_action.east) -- node[above] {yes} (execute);
    \path[line] (global_path.west) -| node[left, below, near start] {no} ([xshift=-3cm, yshift=8.31cm]global_path.west) -|  (subtasks.south); 
   
    \draw[-stealth] (first_action.north east) -- node[near end, right] {no} ([xshift=1.7cm, yshift=0.39cm]first_action.north) |- ([yshift=-0.35cm]single_multi.south) -- (single_multi.south);
    \draw [draw=white,double=black,double distance=\pgflinewidth,ultra thick] (manipulation_planning.east) -- +(1cm,0);
    \draw [draw=white,double=black,double distance=\pgflinewidth,ultra thick] (manipulation_planning.north) -- +(0,1cm);
    
    \draw[-stealth] ([yshift=0.2cm, xshift=0.2cm]execute.south east) --  ([yshift=-0.8cm, xshift=1.2cm]execute.south east) -- node[at end, left] {robot input, action feedback} +(0,-2.7cm);
    
    \draw[stealth-] ([yshift=-0.2cm, xshift=-0.2cm]execute.south east) --  ([yshift=-1.2cm, xshift=0.8cm]execute.south east) -- node[left, at end] {sensor measurements} +(0, -1.8cm);
    
    \path[line] (execute.south) |- node[near start, left] {success} (target_node_reached.east);
    \draw[-stealth] (execute.east) -- node[above] {failure} ([xshift=1.5cm]execute.east) |- (path_existence.east);
    
    
    % edges, Target node reached 
    \path[line] (target_node_reached.north) -- node[left] {no} (global_path.south);
    \path[line] (target_node_reached.west) -- node[above] {yes} (end.east);
\end{tikzpicture}
\caption{Flowchart displaying the hypothesis graph's workflow.}
\label{tikz:flowchart_hgraph}% 
\end{figure}

\restoregeometry

\begin{figure}[H]
    \centering
    \includegraphics[width=5cm]{figures/boxer_robot.png}
    \caption{Pushing task through blocked corridor with the point robot, a green cube to push toward the target ghost state and a red blockade.}
    \label{fig:blocked_path_example_environment}
\end{figure}

\begin{figure}[H]
    \centering
    \begin{subfigure}{.5\textwidth}
    \centering
    \includegraphics[width=0.8\textwidth]{figures/boxer_robot.png}
    \caption{todo}
    \label{subfig:todo}
    \end{subfigure}%

    \begin{subfigure}{.5\textwidth}
    \centering
    \includegraphics[width=0.8\textwidth]{figures/boxer_robot.png}
    \caption{todo}
    \label{subfig:B}
    \end{subfigure}
    \caption{todo}
    \label{subfig:blocked_path_hgraph_exmple}
\end{figure}

\todo[inline]{update the figure above here, Martijn did not like single/multi body, completely replace these terms.}
\todo[inline]{make the colors different, some which can be visualised with the airlab monitors.}

\todo[inline]{here are some example hgraph's required}

\todo[inline]{should I do an example Hgraph here? that requires target ghost positions. yes implement a }

As can be seen in \cref{tikz:flowchart_hgraph} there are some methods used which are still unexplained. Such as path non-existence, or motion planning. 
\todo[inline]{walk through the flowchart, what is actually happening here, It might be in your mind, but can the reader understand the flowchart without any additional context? clarify how edges are initialised, hypotheses are formed and how replanning occurs}


\subsection{Example}%
\label{subsec:hgraph_example}

\todo[inline]{here are some example hgraph's required}
\begin{figure}[H]
    \centering
    \includegraphics[width=5cm]{figures/boxer_robot.png}
    \caption{Pushing task through blocked corridor with the point robot, a green cube to push toward the target ghost state and a red blockade.}%
\label{fig:blocked_path_example_environment}
\end{figure}

\begin{figure}[H]
    \centering
    \begin{subfigure}{.5\textwidth}
    \centering
    \includegraphics[width=0.8\textwidth]{figures/boxer_robot.png}
    \caption{todo}%
    \label{subfig:todo}
    \end{subfigure}%

    \begin{subfigure}{.5\textwidth}
    \centering
    \includegraphics[width=0.8\textwidth]{figures/boxer_robot.png}
    \caption{todo}%
    \label{subfig:B}
    \end{subfigure}
    \caption{todo}%
    \label{subfig:blocked_path_hgraph_exmple}
\end{figure}




\input{mainmatter/hypothesis_graph/h-algorithm}
\section{Knowledge Graph}%
\label{sec:kgraph}
The \ac{hgraph} discussed in previous section has a lifetime that spans over a single task, learned system models are not stored for \ac{hgraph} that are created for future tasks. Storing learned environment knowledge is the \ac{kgraph}'s responsibility. Another responsibility of the \ac{kgraph} is to make an ordering in the stored environment knowledge. The ordering is made with a proposed success factor, a metric that combines multiple metrics such as prediction error, tracking error and the success-fail ratio of a edge parameterisation (controller and system model). 

\todo[inline]{explainer of the name \ac{kgraph}}

\subsection{Definition}
\label{subsec:kgraph_definition}


\todo[inline]{this section}

\subsection{Edge Metrics}
\label{subsec:edge_metrics}
\subsection{Example}
\label{subsec:kgraph_example}



\subsection{Example}%
\label{subsec:kgraph_example}
An example \ac{kgraph} can be visualized in \Cref{fig:kgraph_example}, the parameterization of edges is displayed and the object that the edge controls as image. For clarification, the connected left part with image of the point robot on the center node has 3 outgoing edges that describe robot driving. The connected part on the right with an image of the point robot and the green box on the center node has 2 outgoing edges that describe robot pushing against the green box.\bs

\begin{figure}[H]
    \centering
    \includegraphics[width=10cm]{figures/kgraph_example}
    \caption{\ac{kgraph} with 3 edges on robot driving, and 2 edges for pushing the green box.}%
    \label{fig:kgraph_example}
\end{figure}

The edges in the figure above display only the edge parameterization, but store more information, mainly the success factor. The blue nodes serve a small purpose, making sure edges can point to a node. The blue nodes could fulfill a larger purpose, that is describing which actuators the edge can control. For example, a mobile robot with robot arm attached can have a set of controllers that only drive the base, a set of controllers that only steer the robot arm and a set that controls both the base and robot arm. In such cases the blue nodes describe which part can of the robot can be actuated. The controllers considered in this thesis control every actuator of the robot, resulting in the blue nodes serving such a small purpose.\bs

\subsection{Edge Metrics}%
\label{subsec:edge_metrics}
The \ac{kgraph} keeps an orderd list of `good' and `bad' edge arguements (controller and system model). `Good' and `bad' are defined by edge metrics, these metrics are created after completion of an edge, regardless of the edge successfully completed or failed. An indication is given on why certain metrics matter in \cref{table:review_edge_metrics}.

\begin{table}[htb!]
\centering
\begin{tabular}[t]{p{3.7cm} p{10cm}}
  \acf{PE}&  To better compare prediction errors the \ac{PE} is summarised and average \ac{PE}. The average \ac{PE} is a indicator of a accurate system model, but can give a misleading results, since \ac{PE} also in an indicator of unexpected collisions. Prediction error should thus only be used if there are not collisions detected. The average \ac{PE} comes with more flaws, since the average is mostly determined by outliers, some unfortunate outliers in the \ac{PE} might for the largest part determine the average \ac{PE}. The average \ac{PE} whill thus not be used because it is not robust enough.\\
\acf{TE}& For a low \ac{TE} the system model must be a close to the real motion equations to yield a feasible path, the controller must be well tuned to be able to track that path and the controller and system model must be in collaboration, because the controller uses the system model to calculate system input. A low \ac{TE} tells multiple things, whilst a high \ac{TE} would indicate improvements could be gained in the controller, the system model or their collaboration.\\
ratio \#succesfully completed edges and \#total edges & Over time the \ac{kgraph} can recommend the same edge arguements multiple times. Loggin the ratio of succeding edges vs total edges builds an evident protofilio. Still this metric has to be taken with a grain of salt, because edges with equal edge arguements perform similar actions e.g.~pushing an object through a wide corridor is compared to pushing the same object through a narrow corridor. One could say: \quotes{comparing apples with pears}.\\
final position and \newline displacement error & The quality of the end result is measured in the final position and displacement error. The importants should thus be stressed when ordering edge arguements.\\
planning time& With system identifiaction, path estimation, motion or manipulation planning the planning time can vary in orders of magnitude between simple or more complex approaches. Planning time mainly serves to rank the slowest planners low, whilst not influencing the rank of fast and average planners.\\
runtime& Also known as execution time, it would be a quality indicator if start and target states would be equal. Edges are recommended to solve similar tasks, where path legnth between start and target state is different. Thus planning time is not of any use to rank edges.\\
completion time = \newline runtime + planning time & With a same arguementation as runtime, completion time is not of any use to rank edges.\\
\end{tabular}
\caption{Edge metrics used to rank control methods from `good' to `bad'}
\label{table:review_edge_metrics}
\end{table}




\todo[inline]{conclusion}




What by now hopefully became clear to the reader is that the \ac{h-algorithm} autonomously searches for hypotheses in the \ac{h-graph} to solve a task, one subtask at a time. The \ac{h-algorithm} switches between the search and execution loop. Switching from the search loop toward the execution loop when a hypothesis is found and switching back when a hypothesis is completed, or a fault is detected.\bs

The limited number of possible edge parameterization (every combination of a system identification method with a compatible control method) guarantees that the robot tries to complete a subtask. However, it concludes that it cannot complete a subtask if all possible edges have failed.\bs

This thesis proposes to combine the three topics (one, learning object dynamics, two, the \ac{NAMO} problem, and three, nonprehensile push manipulation to target pose). The \ac{h-algorithm} can solve \ac{NAMO} problems because the robot can drive toward target poses even if reaching such a pose requires objects to be moved first. The proposed algorithm learns to classify objects by updating the object's class from unknown to movable or unmovable. The \ac{h-algorithm} can push objects to target poses by identifying a system model and then pushing the object toward its target pose. However, the system model that system identification yields is of short use because it is only given to the corresponding action edge.\bs

