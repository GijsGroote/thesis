\subsection{Definition}%
\label{subsec:kgraph_definition}
The responsibility of the \ac{kgraph} is to store object class information and to collect edge feedback, to then suggest edge parameterization based on the collected feedback. Estimating which parameterization would be the best candidate is an entire field of research. In this thesis, a simple metric, the success factor, has been chosen based on the average prediction error and the number of times an edge succeeded or failed. From the point of the \ac{kgraph} there is little information to work with; feedback must be created with only information on prediction error, tracking error and whether a fault was detected. Then action suggestions must be made based on collected feedback and an object that should change the start configuration to a target configuration (connecting two nodes in the \ac{hgraph}). A simple success factor thus already incorporates most of the available metrics. First, the definition of the edges, center-, side nodes and success factor are given. Then the \ac{kgraph} is defined.\bs

The \ac{kgraph} consists of center nodes, side nodes and edges. A center node's responsibility is to store an object that corresponds to an object in the robot environment and its classification.\bs

Formally, a \textbf{center node}, $\gls{node}^{\mathit{center}}_{id} =\left\langle \mathit{id}, \mathit{id_{obj}}, \gls{objectClass} \right\rangle $\bs

Where \textit{id} an identifier for the center node, $\mathit{id_{obj}}$ an identifier for the object, \gls{objectClass} the classification of that object than can be either MOVABLE or UNMOVABLE.\bs

\noindent A side node is a placeholder for the edge to point to.\bs

Formally, a \textbf{side node}, $\gls{node}^{\mathit{side}}_{id} =\left\langle id \right\rangle $\bs

\noindent An edge describes its parameterization and how that parameterization compares to other edges in the \ac{kgraph}.\bs

Formally an \textbf{edge}, $\gls{edge}_{(from, to)} = \left\langle id_{from}, id_{to}, \gls{successfactor}(a), \textrm{System Model}, \textrm{Controller}\right\rangle$\bs

Where (System Model, Controller) together is referred to as edge parameterization. An center node's task is to represent an object and store its class which can be Movable or Unmovable.\bs

The successfactor rates an edge parameterization based on experience in the robot environment. The successfactor is created or updated when an action edge with that parameterization failed during execution time or successfully completed.\bs

Formally the \textbf{success factor} = \gls{successfactor}:
\[\gls{successfactor}(a+1) =
  \begin{cases} 0.1^{\gls{pe}_\textrm{avg}}& \textrm{if $a$ = 0}\\[5px]
    0.1 + 0.9\gls{successfactor}(a) & \textrm{if $a > 0$ and the reviewed edge was successfully completed}\\[5px]
  0.9\gls{successfactor}(a) & \textrm{if $a > 0$ and the reviewed edge failed during execution time}
\end{cases}\]

Where $\gls{pe}_\textrm{avg}$ is the average \ac{PE} of the action edge that is reviewed. $a$ is the number of times an action edge with such a edge parameterization reached the EXECUTING status.\bs

\noindent Now that the nodes and edges have been defined, the \ac{kgraph} can be defined.\bs

Formally, a \textbf{\acl{kgraph}}, $\gls{kgraph} = \left\langle \gls{nodesK}, \gls{edgesK} \right\rangle $
\\comprising $\gls{nodesK} = \{\gls{node}^{\mathit{center}}, \gls{node}^{\mathit{side}}\}$, \quad $\gls{edgesK} \in \{\gls{edge}_{(i,j)}| i \in \gls{nodesK}^\mathit{center}_\mathit{ids}, j \in \gls{nodesK}^\mathit{side}_\mathit{ids} \}$.\bs

Where $\gls{nodesK}^\mathit{center}_\mathit{ids}$ are the identifiers of the set of center nodes, and $\gls{nodesK}^\mathit{side}_\mathit{ids}$ are the identifiers of the set of side edges.\bs

When the \ac{halgorithm} determined is must manipulate an object to fullfill the given task, it initally creates a random edge parameterization. That parameterization is used to manipulate the object and stored in the \ac{kgraph}. When encountering the need to manipulate the object again a different random parameterization is selected. First, all possible edge parameterizations are tested on a specific object, this can be seen as the testing phase of an object. Then second, when all possible parameterizations are tested, the \ac{kgraph} suggests the edge parameterization with the highest success factor, this phase can be seen as the converging phase. Where either, the edge parameterization with the highest success factor receives an even higher success factor, or the success factor is lowered, and the second best edge parameterization becomes the highest ranking edge parameterization in terms of success factor.\bs


The \ac{kgraph} has three interface functions. The \textit{add\_object} function adds object information to the \ac{kgraph}, which is important for adding unmovable obstacles that the robot cannot manipulate. The \textit{add\_review} function is used when an edge is successfully or unsuccessfully completed, and the corresponding node in the \ac{kgraph} is updated with a new success factor as described in the formula above. The \textit{action\_suggestion} returns the best parameterization it contains for an object.\bs
