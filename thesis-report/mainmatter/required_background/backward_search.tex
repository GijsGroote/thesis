% COMMENT FROM GIJS 
% This backward search is a nice proposition which could be build on top of the proposed method, perhaps, move it toward  future work



%
\section{Backward Search}%
\label{sec:backward_search}
TODO hre GIssa FISSA
This section introduces the \ac{RS} algorithm, an algorithm capable of finding a feasible order in which to rearrange objects to a target configuration~\cite{krontiris_dealing_2015}. In upcoming chapther the proposed method will rely upon this \ac{RS} algorithm to find a feasible order to complete a specified task. The \ac{RS} algorithm is an existing algorithm and has been proven and tested in 2015 by \citeauthor{krontiris_dealing_2015}. The rearrangement search algorithm relies on a technique known as \textbf{backward induction} also known as \textbf{backward tracing} or \textbf{backward search}. This thesis will refer to the underlying technique of the \ac{RS} algorithm with the term \quotes{backward search}. \citeauthor{krontiris_dealing_2015} proposed two versions of the rearrangement algorithm the \ac{mRS} algoritm that specilises in monotone challenges where objects can only be rearrange once, and the \ac{plRS} that specialises in nonmonotone challenges which result in searching in a piecewise-linear configuration space. In this thesis the focus lies upon \ac{plRS}, and a from this point the \ac{RS} algorithm refers to the \ac{plRS} algorithm.\bs
%
% Now the \ac{plRS} algorithm will be briefly discussed. 
% \todo[inline]{explain how a backward search works, but keep it compact, the interested readers should go find the referce}
