\section{System Identification}%
\label{sec:sys_iden}
Understanding the world is captured by models, \textit{system models} that estimate the behavior of real-world systems. For example what trajectory does an object take after receiving a push? The trajectory is dependent on the properties of the object (e.g.~mass, geometry), and interaction with its surroundings (e.g sliding friction). These properties and interactions are captured by a system model. Just as applicable constraints that are captured by a system model, see both robots in \Cref{fig:example_robots}, the holonomic point robot in \Cref{subfig:example_point_robot} can drive without constraints, and the boxer robot in \Cref{subfig:example_boxer_robot} can drive forward, backwards and can rotate. A system model for robot driving for the boxer robot should thus capture that it is not able to drive directly sideways.\bs

This thesis encounters only 2 different robots but many different objects. Since every object can be different in type, weight and dimensions. System identification methods are required to capture the variety of objects that the robot can encounter.\bs

\todo[inline]{Visualise test push sequence to collect IO data, once with the robot to generate a system model for drive applications}
\todo[inline]{Visualise test push sequence to collect IO data, once with the robot and object to generate a system model for push applications}
\todo[inline]{Paragraph on the input that persistently excites the system.}
