\subsection{Definition}%
\label{subsec:hgraph_definition}%
Before defining the \ac{hgraph}, some definitions are defined on which the \ac{hgraph} depends. First, recall the \textbf{configuration} defined in the \Cref{sec:problem_description}.\bs

Formally a \textbf{configuration}, $\gls{c}_{id}(k)$ is a tuple of $\left\langle \gls{x}(\gls{k}), \gls{y}(\gls{k}), \gls{theta}(\gls{k})\right\rangle$\\ where $\gls{x}, \gls{y} \in \mathbb{R}, \quad  \gls{theta} \in [0, 2\pi)$\bs

An object holds the information about an objects configuration and shape of the object.\\Formally, a \textbf{object},  $\gls{obj}_{id}(\gls{k}) = \left\langle \gls{c}(k), \mathit{shape} \right\rangle $\bs

where $\mathit{shape}$ is linked to a 3D representation of the object, \textit{id} is an identifier for the object.\bs

An object node represents an object in a configuration.\\Formally, a \textbf{objectNode}, $V^{\gls{obj}}_{id} =\left\langle \textrm{status}, \gls{obj}(k)\right\rangle $.\bs

An edge describes the details of how a node transitions to another node in the \ac{hgraph}. In the robot environment, an edge represents a change of configuration of an object. Edges are split into 2 categories, system identification edges that have as goal to collect \ac{IO} data and generate a system model, and action edges that steer a system toward a target configuration. The different goals make action and system identification edges very different, which is why the distinction was make. At the same time the edges both represent a change in the environment as a result of the robot driving, pushing or collecting \ac{IO} data. The edges are formally defined as:\bs

A \textbf{identification edge}, \[\gls{edge}_{(from, to)} = \left\langle \textrm{status}, id_{from}, id_{to}, \textrm{Identification Method}, \textrm{controller}, \textrm{input}\right\rangle\]\bs

With $id_{from}$ and $id_{to}$ indicating the node identifier of the node in the \ac{hgraph} where the edge start from and point towards respectively, identification method indicates the used method that converts \ac{IO} data to a system model, the controller contains the control method used for driving the robot during the collection of \ac{IO} data and input contains multiple sequences of input to send toward the system. For an identification edge it is important that it is compatible with the controller residing in the corresponding action edge.\bs

A \textbf{action edge}, \[\gls{edge}_{(from, to)} = \left\langle \textrm{status}, id_{from}, id_{to}, \textrm{verb}, \textrm{controller},\textrm{dynamic model}, \textrm{path}\right\rangle\]\bs

With $id_{from}$ and $id_{to}$ indicating the node identifier of the node in the \ac{hgraph} where the edge start from and point towards respectively, verb an English verb describing the action the edge represents (driving, pushing), the controller contains the control method used for driving the robot, the dynamic model is the dynamic model used by the control method and the path a list of configurations indicating the path connecting a start- to target node.\bs

Now the nodes and edges have been defined, the \ac{hgraph} can be defined.\bs

Formally, a \textbf{hypothesis graph}, $\gls{hgraph} = \left\langle \gls{nodesH}, \gls{edgesH} \right\rangle $
\\comprising $\gls{nodesH} = \{\gls{nodesH}^{\gls{obj}}_{i}\}$, \quad $\gls{edgesH} \in \{\gls{edge}_{(i,j)}| \gls{edgesH}_i, \gls{edgesH}_j \in \{\gls{nodesH}^{\gls{obj}} \}, i \neq j\}$.\bs

Most \ac{hgraph} components have been defined. The status of an identification edge or action edge still remains undefined and requires some further explanation.\bs

\paragraph{Status, Types and Lifetime of edges}
System identification and tracking a path are so very different, the edges are split into two categories, identification edges and action edges. An identification edge, which is responsible for sending an input sequence to the system and recording the system output. That \ac{IO} sequence and assumptions on the system are the basis for system identification, techniques on various system identification methods are discussed in \Cref{sec:sys_iden}. The goal is to create a dynamical model which is augmented with a corresponding controller is closed-loop stable. The status of an identification edge can be visualized in the following \ac{FSM}.\bs

\begin{figure}[H]
\centering
\begin{tikzpicture}[node distance = 2cm, auto, initial]
    \node [state, fill=my_dark_blue] (init_test_num) {PrIS\#t};
    \node [state, fill=my_light_blue, below of=init_test_num] (completed_test_num) {PoIS\#t};
    \node [state, accepting, fill=my_green, below of=completed_test_num] (completed) {CO};
    \node [state, accepting, fill=my_red, right of=completed_test_num, node distance=6cm] (failed) {FAIL};

 % arrows
    \draw [-stealth] ([xshift=-2cm]init_test_num.west) to node[near start,above]{\shortstack[]{Select compatible\\sys. iden. method\\Go to first $\gls{c}_{\mathit{start}}$}} (init_test_num.west);
    \draw[-stealth] (init_test_num) edge[bend right] node[left]{Collect \ac{IO} data} (completed_test_num)
(completed_test_num) edge node[left]{Create system model} (completed);
    \draw [-stealth] (completed_test_num) edge[bend right] node[right]{\shortstack{Go to next\\start configuration}} (init_test_num);
    \draw [-stealth] (completed_test_num) to node[below]{\shortstack[]{Unable to reach next\\start configuration}}  (failed.west);
    % \draw [-stealth] (init_test_num) [out=0, in=90] to node[above]{Unable to reach starting robot configuration}  (failed.north);

\end{tikzpicture}
\caption{\acs{FSM} displaying the status of an identification edge}%
\label{tikz:status_identification_edge}
\end{figure}

\noindent
\begin{table}[H]
\centering
\begin{tabular}%
  {>{\raggedleft\arraybackslash}p{0.30\textwidth}%
   >{\raggedright\arraybackslash}p{0.60\textwidth}}
PRE INPUT SEQUENCE number t (PrIS\#t): & Go to target configuration to input the input sequence on the system. \\
POST INPUT SEQUENCE number t (PoIS\#t): & Collect the output sequence. \\
COMPLETED (CO): & The edge has driven the system toward its target configuration and its performance has been calculated. \\
FAILED (FAIL): & An error occurred, yielding the edge unusable. \\
\end{tabular}
\end{table}

An identification edge corresponds to an action edge, because its goal is to generate an system model to then hand over its corresponding action edge. The system identification method is selected after the action edge is selected such that the system identification method yields a system model compatible with the controller that resides in the action edge. Two types of system models generated, system models that describe the driving behavior of the robot, and system models that describe the push behavior of the robot and an object. Data collection for a driving model can be collected by sending input to the robot assuming that the robot has enough free space around it such that is does not collide with an object during data collection. To collect \ac{IO} data for a pushing model the robot first drives to a starting configuration next to the object. After the first input sequence is over, the robot drives to the next start position next to the object. After several input sequences the robot has gathered enough \ac{IO} data to generate a system model for pushing.\bs

An action edge, contains a drive or push action. From the point of initialization an action edge starts collecting all necessary ingredients such that is can be executed (track a path). These ingredients are estimating path existence, being given a system model and searching for a path. To collect such ingredients the edge performs path estimation, is given a system model, performs motion or manipulation planning. Then finally, the edge is ready to be executed and send input toward the robot, an \ac{FSM} of the action edge's status can be visualized in \Cref{tikz:status_action_edge}.\bs

\begin{figure}[H]
\centering
\begin{tikzpicture}[node distance = 2cm, auto, initial]
    \node [state, fill=my_purple] (init) {IN};
    \node [state, fill=my_dark_blue, below of=init] (path_exist) {PE};
    \node [state, fill=my_light_blue, below of=path_exist] (system_model) {SM};
    \node [state, fill=my_green, below of=system_model] (path_planned) {PP};
    \node [state, fill=my_yellow, below of=path_planned] (executing) {EX};
    \node [state, accepting, fill=my_orange, below of=executing] (completed) {CO};
    \node [state, accepting, fill=my_red] (failed) at ([xshift=4cm]$(system_model)!0.5!(path_planned)$) {FAIL};

 % arrows
    \draw [-stealth] ([xshift=-2cm]init.west) to node[near start,above]{Select controller} (init.west);
    \draw[-stealth] (init) edge node[left]{Graph-based path estimation} (path_exist)
      (path_exist) edge[bend right] node[left]{Load in system model} (system_model)
(system_model) edge[bend right] node[left]{Motion planning} (path_planned)
(path_planned) edge[bend right] node[left]{Go to execution loop} (executing)
(executing) edge[bend right] node[left]{Completed} (completed);

    \draw [-stealth] (init.east) [out=0, in=90] to node[xshift=0.1cm, right]{Path non-existence proven}  ([yshift=-0.03cm,xshift=0.2cm]failed.north);
    \draw [-stealth] (path_exist.east) [out=0, in=90] to node[xshift=-0.6cm,yshift=0.55cm, above]{\shortstack[l]{System\\identification\\error}}  ([yshift=-0.03cm,xshift=-0.2cm]failed.north);
    \draw [-stealth] (system_model.east) [out=0, in=180] to node[xshift=0.1cm, yshift=0.3cm, above]{\shortstack[l]{Motion\\planning\\error}} (failed.west);
    node[right]{motion planning error}
    ([yshift=-0.3cm]failed.west);
    \draw [-stealth] (executing.east) [out=0, in=-90] to node[xshift=0.1cm,right]{Fault detected}(failed.south);

\end{tikzpicture}
\caption{\acs{FSM} displaying the status of an action edge}%
\label{tikz:status_action_edge}
\end{figure}

\noindent
\begin{table}[H]
\centering
\begin{tabular}%
  {>{\raggedleft\arraybackslash}p{0.24\textwidth}%
   >{\raggedright\arraybackslash}p{0.66\textwidth}}
INITIALIZED (IN): & The edge is created with a source and target node which are present in the \ac{hgraph}. A choice of controller is made. \\
PATH EXISTS (PE): & A graph-based search is performed to validate if the target configuration is reachable assuming that the system is holonomic. \\
SYSTEM MODEL (SM): & A dynamics system model is provided to the controller residing in the edge. \\
PATH PLANNED (PP): & Resulting from a sample-based planner, a path from start to target configuration is provided. \\
EXECUTING (EX): & The edge is currently receiving observations from the robot environment and sends back robot input. \\
COMPLETED (CO): & The edge has driven the system toward its target configuration and its performance has been calculated. \\
FAILED (FAIL): & An error occurred, yielding the edge unusable. \\
\end{tabular}
\end{table}

\Cref{tikz:status_action_edge} shows that many steps must successfully be completed before the edge can be executed. 
Before executing edges, edges must be initialized, which is where next section is dedicated to.\bs
