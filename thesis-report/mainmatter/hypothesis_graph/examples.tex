\subsection{Examples}%
\label{subsec:hgraph_example}

Before displaying example \ac{hgraph}'s, a legend is presented below.\bs

\begin{figure}[H]
    \centering
    \begin{subfigure}{0.2\textwidth}
    \centering
    \includegraphics[width=0.7\textwidth]{figures/proposed_method/connecting_nodes/legend/node}
    \caption{Regular node created by the \ac{halgorithm}.\newline}%
    \end{subfigure}
    \begin{subfigure}{0.2\textwidth}
    \centering
    \includegraphics[width=0.7\textwidth]{figures/proposed_method/connecting_nodes/legend/current_node}
    \caption{Current node indicates that it's outgoing edge is or is next to be executed.}%
    \end{subfigure}
    \begin{subfigure}{0.2\textwidth}
    \centering
    \includegraphics[width=0.7\textwidth]{figures/proposed_method/connecting_nodes/legend/starting_node}
    \caption{Starting node, one is generated at for every subtask.}%
    \end{subfigure}
    \begin{subfigure}{0.2\textwidth}
    \centering
    \includegraphics[width=0.7\textwidth]{figures/proposed_method/connecting_nodes/legend/target_node}
    \caption{Target node, one is generated for every subtask.\newline}%
    \end{subfigure}

    \begin{subfigure}{0.33\textwidth}
    \centering
    \includegraphics[width=0.7\textwidth]{figures/proposed_method/connecting_nodes/legend/edge}
    \caption{Edge with status IN, PE, SM, PP or EX.}%
    \end{subfigure}
    \begin{subfigure}{0.33\textwidth}
    \centering
    \includegraphics[width=0.7\textwidth]{figures/proposed_method/connecting_nodes/legend/failed_edge}
    \caption{Edge with status FAILED (FAIL)}%
    \end{subfigure}
    \begin{subfigure}{0.33\textwidth}
    \centering
    \includegraphics[width=0.7\textwidth]{figures/proposed_method/connecting_nodes/legend/completed_edge}
    \caption{Edge with status COMPLETED (CO)}%
    \end{subfigure}
    \caption{Legend for \ac{hgraph}'s nodes an edges}%
    \label{fig:hgraph_legend}
\end{figure}

\paragraph{Driving and Pushing} Four examples are presented, starting with a driving task in \Cref{fig:robot_drive_hgraph}, then a pushing task in \Cref{fig:robot_push_hgraph}.\bs

\begin{figure}[H]
    \centering
    \begin{subfigure}{.3\textwidth}
    \centering
    \includegraphics[width=0.7\textwidth]{figures/proposed_method/connecting_nodes/robot_to_target/robot_to_target}
    \end{subfigure}
    \begin{subfigure}{.3\textwidth}
    \centering
    \includegraphics[width=0.9\textwidth]{figures/proposed_method/connecting_nodes/robot_to_target/robot_drive_target}
    \end{subfigure}
    \begin{subfigure}{.3\textwidth}
    \centering
    \includegraphics[width=\textwidth]{figures/proposed_method/connecting_nodes/robot_to_target/robot_iden_drive_target}
    \end{subfigure}
    \caption{\ac{hgraph} generated by the \ac{halgorithm} to drive the robot to a target configuration}%
    \label{fig:robot_drive_hgraph}
\end{figure}

The robot does not have a system model of itself, thus first system identification must be performed before it can drive to the specified target configuration. The \ac{kgraph} that will be discussed in \Cref{subsec:kgraph_definition} can suggest an action that includes a system model. In that case, system identification is not needed. The following figure displays successfully executing the hypothesis found in \Cref{fig:robot_drive_hgraph}.\bs

\begin{figure}[H]
    \centering
    \begin{subfigure}{.3\textwidth}
    \centering
    \includegraphics[width=0.9\textwidth]{figures/proposed_method/connecting_nodes/robot_to_target/execute_robot_to_target_1}
    \end{subfigure}
    \begin{subfigure}{.3\textwidth}
    \centering
    \includegraphics[width=0.9\textwidth]{figures/proposed_method/connecting_nodes/robot_to_target/execute_robot_to_target_2}
    \end{subfigure}
    \begin{subfigure}{.3\textwidth}
    \centering
    \includegraphics[width=0.9\textwidth]{figures/proposed_method/connecting_nodes/robot_to_target/execute_robot_to_target_3}
    \end{subfigure}
    \caption{Executing the hypothesis found in \Cref{fig:robot_drive_hgraph}.}
    \label{fig:execute_robot_to_target}
\end{figure}

Upcoming figure will display the hypothesis generated to push an object to a target position. Both generating a hypothesis and executing the hypothesis are intertwined, this is because certain information should first be collected from the environment before the full hypothesis can be generated. An example is the \textit{best\_push\_position} that can be found in \Cref{subfig:robot_push_7,subfig:robot_push_8,subfig:robot_push_9}. The \textit{best\_push\_position} can be found after manipulation planning for the pushing edge is completed. For motion planning a system model is required, thus the corresponding system identification edge should be completed before manipulation planning can start, and than the \textit{best\_push\_position} can be determined.\bs

\begin{figure}[H]
    \centering
    \begin{subfigure}{.3\textwidth}
    \centering
    \includegraphics[width=0.8\textwidth]{figures/proposed_method/connecting_nodes/robot_push/robot_push_1}
    \caption{}
    \end{subfigure}
    \begin{subfigure}{.3\textwidth}
    \centering
    \includegraphics[width=1.1\textwidth]{figures/proposed_method/connecting_nodes/robot_push/robot_push_2}
    \caption{}\label{subfig:robot_push_2}
    \end{subfigure}
    \begin{subfigure}{.3\textwidth}
    \centering
    \includegraphics[width=1\textwidth]{figures/proposed_method/connecting_nodes/robot_push/robot_push_3}
    \caption{}\label{subfig:robot_push_3}
    \end{subfigure}

    \begin{subfigure}{.3\textwidth}
    \centering
    \includegraphics[width=1\textwidth]{figures/proposed_method/connecting_nodes/robot_push/robot_push_4}
    \caption{}\label{subfig:robot_push_4}
    \end{subfigure}
    \begin{subfigure}{.3\textwidth}
    \centering
    \includegraphics[width=1.05\textwidth]{figures/proposed_method/connecting_nodes/robot_push/robot_push_5}
    \caption{}\label{subfig:robot_push_5}
    \end{subfigure}
    \begin{subfigure}{.3\textwidth}
    \centering
    \includegraphics[width=1.05\textwidth]{figures/proposed_method/connecting_nodes/robot_push/robot_push_6}
    \caption{}\label{subfig:robot_push_6}
    \end{subfigure}

    \begin{subfigure}{.3\textwidth}
    \centering
    \includegraphics[width=1\textwidth]{figures/proposed_method/connecting_nodes/robot_push/robot_push_7}
    \caption{}\label{subfig:robot_push_7}
    \end{subfigure}
    \begin{subfigure}{.3\textwidth}
    \centering
    \includegraphics[width=1.05\textwidth]{figures/proposed_method/connecting_nodes/robot_push/robot_push_8}
    \caption{}\label{subfig:robot_push_8}
    \end{subfigure}
    \begin{subfigure}{.3\textwidth}
    \centering
    \includegraphics[width=1.05\textwidth]{figures/proposed_method/connecting_nodes/robot_push/robot_push_9}
    \caption{}\label{subfig:robot_push_9}
    \end{subfigure}
    \caption{\ac{hgraph} for pushing the green box to the target configuration}%
    \label{fig:robot_push_hgraph}
\end{figure}
Especially in \Cref{subfig:robot_push_2,subfig:robot_push_3} the backward search is clearly visible, the \ac{halgorithm} searches from target node to the robot node. \Cref{fig:robot_push_hgraph} is extensive because every necessary 

\todo{Corrado: bit unclear here, rewrite, IDK what's happenign here right}
steps is included whilst some could be skipped. First, identifying a system model for robot driving twice, if the system model created in edge Sys. Iden.~pointing toward node robot\_model is reused, then the edge Sys. Iden.~pointing toward robot\_model\_1 would be unnecessary. Second, if system models would already be available for driving and pushing, no single system identification edge would be required. A \textit{empty\_edge} can be seen in \Cref{subfig:robot_push_7,subfig:robot_push_8,subfig:robot_push_9}, the empty\_edge serves to connect a node to another node (box\_model to 

\todo{Corrado: why is there a robot copy?}
robot\_copy in \Cref{fig:robot_push_hgraph}). The empty\_edge can be traversed without execution, holds no controller, system model or status.\bs

\paragraph{Encountering a Blocked Path}%
During propagation of an action edge's status, motion or manipulation planning occurs. If an object is blocking the path, planning will detect it and the \ac{halgorithm} tries to free the path. In the next example the \ac{halgorithm} detects a blocking object and frees the path by pushing the blocking object to a new configuration, and can be visualized in \Cref{fig:blocking_obj_hgraph}.\bs

\begin{figure}[H]
    \centering
    \begin{subfigure}{.3\textwidth}
    \centering
    \includegraphics[width=0.6\textwidth]{figures/proposed_method/connecting_nodes/blocking_obj/blocking_obj_1}
    \caption{}
    \end{subfigure}
    \begin{subfigure}{.3\textwidth}
    \centering
    \includegraphics[width=0.9\textwidth]{figures/proposed_method/connecting_nodes/blocking_obj/blocking_obj_2}
    \caption{}\label{subfig:blocking_obj_2}
    \end{subfigure}
    \begin{subfigure}{.3\textwidth}
    \centering
    \includegraphics[width=\textwidth]{figures/proposed_method/connecting_nodes/blocking_obj/blocking_obj_3}
    \caption{}\label{subfig:blocking_obj_3}
    \end{subfigure}

    \begin{subfigure}{.3\textwidth}
    \centering
    \includegraphics[width=1.291\textwidth]{figures/proposed_method/connecting_nodes/blocking_obj/blocking_obj_4}
    \caption{}\label{subfig:blocking_obj_4}
    \end{subfigure}
    \begin{subfigure}{.3\textwidth}
    \centering
    \includegraphics[width=\textwidth]{figures/proposed_method/connecting_nodes/blocking_obj/blocking_obj_5}
    \caption{}\label{subfig:blocking_obj_5}
    \end{subfigure}
    \begin{subfigure}{.3\textwidth}
    \centering
    \includegraphics[width=\textwidth]{figures/proposed_method/connecting_nodes/blocking_obj/blocking_obj_6}
    \caption{}\label{subfig:blocking_obj_6}
    \end{subfigure}
    \caption{\ac{hgraph} for driving to target configuration and encountering a blocked path}%
    \label{fig:blocking_obj_hgraph}
\end{figure}

\paragraph{Encountering Failure}%
In the last example, \Cref{fig:failure_in_hgraph} the first hypothesis fails to complete and the \ac{halgorithm} tries to generate a new hypothesis that also fails to complete. Several faults and failures are modeled \todo{Corrado: where please ref?}, the \ac{halgorithm} response to faults and failure is the same. If during the propagation of an edge's status any kind of failure arises, the failed edge and corresponding edges are marked as failed. Equally during execution, if a fault is detected, the execution halts and the edge and corresponding edges are marked as \quotes{failed}, the procedure can be seen in \Cref{fig:failure_in_hgraph}.\bs

\begin{figure}[H]
    \centering
    \begin{subfigure}{.3\textwidth}
    \centering
    \includegraphics[width=0.8\textwidth]{figures/proposed_method/connecting_nodes/failure/fail_1}
    \end{subfigure}
    \begin{subfigure}{.3\textwidth}
    \centering
    \includegraphics[width=1.05\textwidth]{figures/proposed_method/connecting_nodes/failure/fail_2}
    \end{subfigure}
    \begin{subfigure}{.3\textwidth}
    \centering
    \includegraphics[width=\textwidth]{figures/proposed_method/connecting_nodes/failure/fail_3}
    \end{subfigure}

    \begin{subfigure}{.3\textwidth}
    \centering
    \includegraphics[width=1\textwidth]{figures/proposed_method/connecting_nodes/failure/fail_4}
    \end{subfigure}
    \begin{subfigure}{.3\textwidth}
    \centering
    \includegraphics[width=1\textwidth]{figures/proposed_method/connecting_nodes/failure/fail_5}
    \end{subfigure}
    \begin{subfigure}{.3\textwidth}
    \centering
    \includegraphics[width=1\textwidth]{figures/proposed_method/connecting_nodes/failure/fail_6}
    \end{subfigure}

    \begin{subfigure}{.3\textwidth}
    \centering
    \includegraphics[width=1\textwidth]{figures/proposed_method/connecting_nodes/failure/fail_7}
    \end{subfigure}
    \hfill
    \caption{Executing two hypothesis, both failing to complete because a fault of failure emerged.}%
    \label{fig:failure_in_hgraph}
\end{figure}

In \Cref{fig:failure_in_hgraph} only two parameterizations of drive controller and system model were available. Thus after two failed hypothesis the \ac{halgorithm} concludes it cannot complete this task.\bs
