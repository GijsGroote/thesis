\section{Hypothesis Graph}%
\label{sec:hgraph}
The \acf{hgraph} consists of a set of nodes and edges. Where the node correspond to an object at a configuration, and the edges correspond to actions. As a whole the \ac{hgraph} represents a search in the composite configuration space. The \ac{halgorithm} creates and updates nodes and edges in the \ac{hgraph} and is discussed in \Cref{subsec:halgorithm}. A search in the composite configuration space is avoided because an edge only operates in a single mode of dynamics, in the scope of this thesis a driving mode or pushing mode. The \ac{hgraph} is created specifically for a task with a single start and a single target node for every subtask in the task. When the \ac{halgorithm} halts and the task is completed, the \ac{hgraph} is no longer needed and is discarded.\bs

In the upcoming section the \ac{hgraph} is defined and discussed in \Cref{subsec:hgraph_definition}. The \ac{halgorithm} is then discussed and in \Cref{subsec:halgorithm}, where an explanation is provided on how the \ac{halgorithm} searches for a solution in the composite configuration space. The section is concluded with an extensive example.\bs

\subsection{Definition}
\label{subsec: hgraph_definition}

\todo[inline]{definition of state, node, }



\textbf{Lifetime and types of edges}
\newline
An edge denotes that input commands are sent to the robot and as a result the robot can drive around changing it's state. Obstacles in a certain state represented by a node in the hgraph can be connected to other nodes via edges.
The edge contains all necessary components to sent input to the robot resulting in an obstacle reaching the target state residing in the node where the edge points toward. \\

There exist two types of edges, identification edges and action edges. An identification edge is responsible for sending an input sequence to the system and recording the system output which the simulation returns. An input/output sequence and assumptions on the system are the basis for system identification techniques discussed in \cref{subsec: sys_iden_and_control_methods}. The goal is to create a dynamical model which if augmented with an corresponding controller is closed-loop stable. \\

An action edge is responsible for putting an obstacle at a target state, the goal is to reach the target state, without triggering the fault detector and scoring well at internal metrics elaborated in \cref{subsec: edge_metrics}.

\todo[inline]{the lifetime of an identification edge}

The different status an edge can have is listed below. An edge status can only change according to the directed arrows in \cref{tikz: lifetime_action_edge}

\definecolor{lavenderIndigo}{RGB}{160, 110, 224}
\definecolor{batteryChargedBlue}{RGB}{22, 164, 216}
\definecolor{skyBlue}{RGB}{96, 219, 232}
\definecolor{kiwi}{RGB}{139, 211, 70}
\definecolor{minionYellow}{RGB}{239, 223, 72}
\definecolor{deepSaffron}{RGB}{249, 165, 44}
\definecolor{sinopia}{RGB}{214, 78, 18}

\begin{figure}[H]
\centering
\begin{tikzpicture}[node distance = 2cm, auto]
    \node [block, fill=lavenderIndigo] (init) {Initialised};
    \node [block, fill=batteryChargedBlue, below of=init] (path_exist) {Path Exists};
    \node [block, fill=skyBlue, below of=path_exist] (system_model) {System Model};
    \node [block, fill=kiwi, below of=system_model] (path_planned) {Path Planned};
    \node [block, fill=minionYellow, below of=path_planned] (executing) {Executing};
    \node [block, fill=deepSaffron, below of=executing] (completed) {Completed};
    \node [block, fill=sinopia] (failed) at ([xshift=4cm]$(system_model)!0.5!(path_planned)$) {Failed};
    
    % arrows
    \draw [-stealth] ([xshift=-2cm]init.west) to node[near start,above]{select controller} (init.west);
    \draw [-stealth] (init.south) to node[left]{graph-based path estimation} (path_exist.north);
    \draw [-stealth] (path_exist.west) [out=215,in=145] to node[left]{system identification} ([yshift=0.3cm] system_model.west);
    \draw [-stealth] ([yshift=-0.3cm] system_model.west) [out=215,in=145] to node[left]{motion planning} ([yshift=0.3cm] path_planned.west);
    \draw [-stealth] ([yshift=-0.3cm] path_planned.west) [out=215,in=145] to  node[left]{go to execution loop} ([yshift=0.3cm] executing.west);
    \draw [-stealth] (executing.south) to node[left]{evaluate with metrics} (completed.north);
    % edges to failed
    \draw [-stealth] (init.east) [out=0, in=90] to node[right]{path non-existence proven}  ([xshift=0.3cm]failed.north);
    \draw [-stealth] (path_exist.east) [out=0, in=90] to node[xshift=-0.4cm,yshift=0.55cm, above]{\shortstack[l]{system\\identification\\error}}  ([xshift=-0.3cm]failed.north);
    \draw [-stealth] (system_model.east) [out=0, in=180] to node[xshift=0.1cm, yshift=0.3cm, above]{\shortstack[l]{motion\\planning\\error}} ([yshift=0.3cm]failed.west);
    node[right]{motion planning error}  
    ([yshift=-0.3cm]failed.west);
    \draw [-stealth] (executing.east) [out=0, in=-90] to node[right]{fault detected}(failed.south);
\end{tikzpicture}
\caption{The status of an action edge and the most important indicator to change status}
\label{tikz: lifetime_action_edge}
\end{figure}

\begin{enumerate}
    \item[INITIALISED] The edge is created with a source and target node which are present in the hypothesis graph. A choice of controller is made.
    \item[PATH EXISTS] A graph-based search is performed to validate if the target state is reachable assuming that the system is holonomic.
    \item[SYSTEM MODEL] A dynamics system model is provided to the controller residing in the edge.
    \item[PATH PLANNED] Resulting from a sampble-based planner, a path from start to target state is provided. 
    \item[EXECUTING] The edge is currently sending input toward the robot. 
    \item[COMPLETED] The edge has driven the system toward it's target state and it's performance has been calculated.
    \item[FAILED] An error occurred, yielding the edge unusable. 
\end{enumerate}
\subsection{Examples}%
\label{subsec:hgraph_example}

Before displaying example \ac{hgraph}'s, a legend is presented below.\bs

\begin{figure}[H]
    \centering
    \begin{subfigure}{0.2\textwidth}
    \centering
    \includegraphics[width=0.7\textwidth]{figures/proposed_method/connecting_nodes/legend/node}
    \caption{Regular node created by the \ac{halgorithm}.\newline}%
    \end{subfigure}
    \begin{subfigure}{0.2\textwidth}
    \centering
    \includegraphics[width=0.7\textwidth]{figures/proposed_method/connecting_nodes/legend/current_node}
    \caption{Current node indicates that it's outgoing edge is or is next to be executed.}%
    \end{subfigure}
    \begin{subfigure}{0.2\textwidth}
    \centering
    \includegraphics[width=0.7\textwidth]{figures/proposed_method/connecting_nodes/legend/starting_node}
    \caption{Starting node, one is generated at for every subtask.}%
    \end{subfigure}
    \begin{subfigure}{0.2\textwidth}
    \centering
    \includegraphics[width=0.7\textwidth]{figures/proposed_method/connecting_nodes/legend/target_node}
    \caption{Target node, one is generated for every subtask.\newline}%
    \end{subfigure}

    \begin{subfigure}{0.33\textwidth}
    \centering
    \includegraphics[width=0.7\textwidth]{figures/proposed_method/connecting_nodes/legend/edge}
    \caption{Edge with status IN, PE, SM, PP or EX.}%
    \end{subfigure}
    \begin{subfigure}{0.33\textwidth}
    \centering
    \includegraphics[width=0.7\textwidth]{figures/proposed_method/connecting_nodes/legend/failed_edge}
    \caption{Edge with status FAILED (FAIL)}%
    \end{subfigure}
    \begin{subfigure}{0.33\textwidth}
    \centering
    \includegraphics[width=0.7\textwidth]{figures/proposed_method/connecting_nodes/legend/completed_edge}
    \caption{Edge with status COMPLETED (CO)}%
    \end{subfigure}
    \caption{Legend for \ac{hgraph}'s nodes an edges}%
    \label{fig:hgraph_legend}
\end{figure}




This chapter has terminology that is conveniently grouped in the following table.

\noindent
\begin{table}[H]
\centering
\begin{tabular}%
  {>{\raggedright\arraybackslash}p{0.23\textwidth}%
   >{\raggedright\arraybackslash}p{0.67\textwidth}}
Task:   &  Tuple of objects and target configurations.\\
        & $\text{task} = \gls{task} = \left\langle \gls{Obj}_{\mathit{task}}, C_{\mathit{targets}} \right\rangle$\\
Subtask:& A single object, and a single target configuration.\\
        & $\text{subtask} = \gls{subtask}= \left\langle \mathit{obj}_{\mathit{subtask}}, \gls{c}_{\mathit{target}} \right\rangle$\\
Object Class & Classification assigned to an object.\\
             & $\gls{objectClass} = \textrm{Unknown}\vee \textrm{Obstacle}\vee \textrm{Movable}$\\
Node Status:& Status of a node indicates if a node is initialized, the \ac{halgorithm} was able to bring the object to the configuration or whether the \ac{halgorithm} fails to bring the object to its configuration.\\
            & \[\gls{nodeStatus} = \mathrm{Initialised} \vee \mathrm{Completed} \vee \mathrm{Failed} \]\\
Node:   & A node in the \acs{hgraph}, represents an object in a configuration with reachability indicated with a node status.\\
        & $\textrm{node} = \gls{node} = \left\langle \textrm{status}, \mathit{obj}, c \right\rangle$\\
Edge Status:& Status of a edge,\\
            & \[\gls{edgeStatus} = \mathrm{Initialised} \vee \mathrm{Path Exists} \vee \mathrm{System Model} \vee \] \[\mathrm{Path Planned} \vee \mathrm{Executing} \vee \mathrm{Completed} \vee \mathrm{Failed}\]\\
  & elaborate information on the edge statuses can be found in \Cref{tikz:status_action_edge}.\\
Edge:   & Edge connecting a node to another node in the \acs{hgraph} or \ac{kgraph}.\\
        & $ \textrm{edge} = \gls{edge} = \left\langle \textrm{status}, id_{\mathit{from}}, id_{\mathit{to}}, \textrm{verb}, \textrm{controller},\textrm{dynamic model}, \textrm{path}\right\rangle$\\
Hypothesis:& Sequence of successive edges in the \ac{hgraph}, an idea to put a object at it's target configuration. If executed and successfully completed, a subtask is completed.\\
           & $ \textrm{hypothesis} = \gls{hypothesis} = \left[\ \gls{edge}_{1}, \gls{edge}_{2}, \gls{edge}_{3}, \dots \gls{edge}_{m} \right]\ $, \hspace{0.5cm} $m>0$\\
Hypothesis Algorithm:& Graph based algorithm that searches for hypothesis in the \ac{hgraph} to complete subtasks eventually completing a task.\\
Hypothesis Graph:& Collection of nodes and edges. For every subtask a start and target node exist in the \ac{hgraph}, the \ac{halgorithm} searches for a path through nodes and edges to connect start to target node.\\
        & $ \textrm{\ac{hgraph}} = \gls{hgraph} = \left\langle \gls{nodesH}, \gls{edgesH} \right\rangle $\\
Knowledge Graph:& Collection of nodes and edges.  The \ac{kgraph} acts as a knowledge base and can be queried for an action suggestion.\\
        & $ \textrm{\ac{kgraph}} = \gls{kgraph} = \left\langle \gls{nodesK}, \gls{edgesK} \right\rangle $\\
\end{tabular}
\caption{Terminology of terms used}
\label{table:proposed_method_terminology}
\end{table}

\todo{add non-failed status}

